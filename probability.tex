\section{Probability}

\subsection{Assiomi di probabilit\'a}

\begin{frame}{Assiomi probabilit\'a Kolmogorov}
\begin{itemize}
\item La probabilit\'a di un evento \'e $\prob{(E)}\geq0$ per ogni $E\in S$ spazio campionario.
\item La probabilit\'a degli eventi del sample space \'e 1.
\item $\sigma$-additivit\'a: per ogni sequenza numerabile di insiemi disgiunti (eventi mutualmente esclusivi)
\begin{equation*}
\prob{(\cup_{i=1}^{\infty}E_i)}=\sum_{i=1}^{\infty}\prob{(E_i)}
\end{equation*}
\end{itemize}
\begin{block}{Addition rule: probability that A or B will}
\begin{equation*}
\prob{(A\cup B)}=\prob{(A)}+\prob{(B)}-\prob{(A\cap B)}
\end{equation*}
\end{block}
\begin{block}{Variabile casuale}
Una variabile che assume uno specifico valore per ogni elemento dello spazio campionario S.
\end{block}
\begin{columns}[T]
\begin{column}{0.5\textwidth}
\begin{block}{Probabilit\'a condizionata}
\begin{equation*}
\prob{(B|A)}=\frac{\prob{(B\cap A)}}{\prob{(A)}}
\end{equation*}
\end{block}
\end{column}
\begin{column}{0.5\textwidth}
\begin{block}{Teorema di Bayes}
\begin{equation*}
\prob{(A|B)}=\frac{\prob{(B|A)}\prob{(A)}}{\prob{(B)}}
\end{equation*}
\end{block}
\end{column}
\end{columns}
\end{frame}

\begin{frame}{Interpretazioni di probabilit\'a}
\begin{block}{Probabilit\'a frequentista}
Gli elementi di S sono risultati di una misura, un sottoinsieme A corrisponde a qualsiasi risultato nel sottoinsieme: A \'e detto evento. Se A contiene solo un elemento \'e detto evento elementare:
\begin{equation*}
\prob{(A)}=\lim_{n\to\infty}\frac{\text{number of occurrences of outcome A in n measurements}}{n}
\end{equation*}
\end{block}
\begin{block}{Probabilit\'a bayesiana }
Gli elementi di A sono ipotesi o affermazioni:
\begin{equation*}
\prob{A}=\text{Degree of belief that A is true}
\end{equation*}
\begin{itemize}
\item Include relative frequency interpretation: The statement that a measure will yield a given outcome a certain fraction of times can be regarded as a hypothesis.
\item Subjective probability can be associated with value of unknown constants (mass of electron, etc): probability of $95\%$ that electron mass is in given interval.
\end{itemize}
\end{block}
\end{frame}

\subsection{Distribuzioni e variabili casuali}

\begin{frame}{Distribuzione di probabilit\'a di un osservabile. Valore di aspettazione, varianza e momenti}
Variabile casuale ($X$) campionata n volte ($x_i$) = Campione di dimensione n
\begin{itemize}
\item Distribuzione di probabilit\'a cumulante: $F(x)=P(x_n<x)$: PDF of X $f(x)=\TDy{x}{F(x)}$.
\item Distribuzione di probabilit\'a di $a(X)$ funzione di variabile casuale X con PDF $f(x)$: $g(a)=f(x(a))|\TDy{a}{x}|$
\item $E[X]=\sum x_iP(X=x_i)$.
\begin{align*}
&E[H(X)]=\intsinf{}H(x)f(x)\,dx
\end{align*}
\item Varianza $\var{(x_n)}=E[(x_n-\mu)^2]=\mu_2=\sigma^2(X)$.
\item momenti: $\mu_l=E[(X-\mu)^l]$
\end{itemize}
\end{frame}

\subsection{Momenti di una distribuzione}

\begin{frame}{Funzione generatrice dei momenti - Funzione caratteristica}

\end{frame}

\begin{wordonframe}{PDF distanza stelle distribuite uniformemente in x,y,z}
\begin{columns}
\begin{column}{0.5\textwidth}
\begin{block}{Radial Fourier transform}
\begin{align*}
&\hat{f}(\vec{k})=\int\exp{-i\scap{k}{x}}f(\vec{x})d^nx\\
&=s^{\frac{n-2}{2}}\hat{F}_n(s)\\
&=(2\pi)^{\frac{n}{2}}\int_0^{+\infty}J_{\frac{n-2}{2}}(sr)r^{\frac{n-2}{2}}F(r)r\,dr
\end{align*}
\end{block}
\end{column}
\begin{column}{0.5\textwidth}
\begin{block}{MGF of sum of RV is product of their MGF}
\begin{align*}
&\MGF{x_i^2}=\E{[\exp{tx_i^2}]}\\
&=\frac{\sqrt{\pi}[\Erfi{(b\sqrt{t})}-\Erfi{a\sqrt{t}}]}{2(b-a)\sqrt{t}}
\end{align*}
\end{block}
\end{column}
\end{columns}
\end{wordonframe}

\begin{wordonframe}{Propriet\'a Trasformata di Fourier}
\begin{columns}
\begin{column}{0.5\textwidth}

\end{column}
\begin{column}{0.5\textwidth}
\begin{align*}
&h(x)=a*f(x)+bg(x)\\
&\hat{h}(\nu)=a\hat{f}(\nu)+b\hat{g}(\nu)\\
&h(x)=\exp{2\pi ix\nu_0}f(x)\\
&\hat{h}(\nu)=\hat{f}(\nu-\nu_0)
\end{align*}
\end{column}
\end{columns}
\end{wordonframe}

\begin{wordonframe}{Trasformata di Fourier notevoli}
\begin{columns}
\begin{column}{0.5\textwidth}
\end{column}
\begin{column}{0.5\textwidth}
\begin{align*}
\F{(\frac{1}{x})}=-i\pi\sgn{(\nu)}(=-i\sqrt{\frac{\pi}{2}}\sgn{(\omega)})
\end{align*}
\end{column}
\end{columns}
\end{wordonframe}