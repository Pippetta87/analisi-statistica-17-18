\section{Probability}

\begin{frame}{assiomi e implementazioni}
\begin{block}{Osservabile sperimentale come variabile stocastica}
Esperimento: procedura definita
\end{block}

\begin{block}{Spazio dei risultati}

\end{block}



\end{frame}

\begin{wordonframe}{Descrizione}

\end{wordonframe}

\section{Compiti}\linkdest{compiti}

\begin{wordonframe}{Combinazione p-value per 10 osservabili indipendenti (16/07/18)}
Fit di massima likelihood per le 10 osservabili (binned) basandosi su teoria: istogrammi suff. popolati per GOF test di Pearson con 30 dof (31 bins?)
\begin{table}[h!]
\centering
\begin{tabular}{||ccccccccccc||} 
$\chi^2(/30?)&38.33&40.87&30.70&36.91&39.97&36.97&20.48&32.34&41.32&41.41$\\
$P(\chi^2_{30}<x)$&0.859&0.911&0.570&0.820&0.894&0.822&0.096&0.648&0.918&0.920\\
\end{tabular}
\end{table}
Impressione che ci siano troppi valori alti del $\chi^2$: come risolvere il dubbio in maniera rigorosa?
\begin{enumerate}
\item \mykeyword{additivit\'a del $\chi^2$}: somma quadrati di RV normali standard. 
\begin{columns}[T]\begin{column}{0.5\textwidth}
Sommando $\chi^2$ in tab si ottiene RV con pdf $\chi^2_{300}$ approssimabile con gaussiana con $\sigma=\sqrt{2N}\approx24.5$.
$\sum\chi^2_i=359.3$, siamo a $1.21\sigma$
\end{column}\begin{column}{0.5\textwidth}
\begin{figure}[!ht]\includegraphics[trim={0cm 0cm 0 0},clip, keepaspectratio,width=\textwidth]{}\caption{}\label{fig:}\end{figure}
\end{column}\end{columns}
\item KS test: campione di 10 valori in accordo con pdf $\chi^2_{30}$ (cumulante $\prob{(\chi^2_{30}<x)}$ in tabella). Calcolo $F_n(x_{k-1})-F_n(x_k)$ e $F_n(x_k)-F_n(x_k)$:  $|0-0.096|$, $|0.1-0.57|$, $|0.2-0.648|$, $|0.3-0.82|$, $|0.4-0.822|$, $|0.5-0.859|$, $|0.6-0.894|$, $|0.7-0.914|$, $|0.8-0.918|$, $|0.9-0.926|$; $\Delta=0.52$??
\item Combinazione diretta p-values: $p_i=1-\prob{(\chi^2_{30}<x)}$, \mykeyword{pdf somma log p-values} $-2\log{(\prod^Np_i)}$ ha pdf $\chi^2_{2N}$
\item Test binned: hist of p-values (3-4) e considero LR con pdf uniforme (\mykeyword{pdf p-value}) e quindi usarlo per test basato su pdf asintotica (esatta, calcolandola)
\item Considerare max(min) dei p-values e usare pdf $x_M$ per test
\item Test della mediana
\item Test basato su propriet\'a pdf $\chi^2_{30}$: test varianza $2N$; test secondo momento attorno a media nominale 30.
\end{enumerate}
\end{wordonframe}

\begin{wordonframe}{Stimatori, confidence/credible interval per parametro m di pdf uniforme da misura della mediana(16/07/18)}
\begin{columns}[T]\begin{column}{0.5\textwidth}
x RV $U(0,m)$; $2k+1=3$ misura la mediana $x_2=3.4$
\begin{block}{MLE estimator, bias, incertezza statistica}
\begin{align*}
&\prob{(\mu=x_2)}=6\frac{\mu}{m^2}(1-\frac{\mu}{m})\\
&\hat{\mu}_{MLE}=3/2\mu\\
&\E{[\hat{m}]}=\frac{3}{2}\E{[\mu]}\frac{3}{2}\int_0^{m}\mu\prob{(\mu)}\,d\mu=\frac{3}{4}m\\
&\hat{m}'=\frac{4}{3}\hat{m}=2\mu\\
&\var{[\hat{m}]}=\sigma^2_{\hat{m}}=\frac{9}{4}\var{\mu}=\frac{9}{4}\sigma_{\mu}^2\\
&\var{[\hat{m}']}=\sigma^2_{\hat{m}'}=4\var{\mu}=4\sigma_{\mu}^2\\
&\sigma^2_{\mu}=\exv{\mu^2}-\exv{\mu}^2=\frac{m^2}{20}
\end{align*}
Le varianze sono funzione del valore di m: le stimo usando $\hat{m}$
\end{block}
\end{column}\begin{column}{0.5\textwidth}
Likelihood pdf uniforme: \begin{equation*}L_x(m)=\left\{\begin{array}{c}
0\ m<x\\
1/m\ m\geq x\\
\end{array}
\end{equation*}
Determinazione $p(\mu=x_2)$ dove $\mu$ pu\'o essere vista come seconda misura pi\'u grande o mediana; per $2k+1$ misure:
\begin{align*}
&\prob{(\mu;m)}=\prob{(\mu)}F(\mu)^k[1-F(\mu)]^k(2k+1)\binom{2k}{k}\\
&\prob{(\mu;m)}=\frac{1}{m}(\frac{\mu}{m})^k(1-\frac{\mu}{m})^k\frac{(2k+1)!}{k!k!1!}
\end{align*}
\begin{block}{\mykeyword{Stima bayesiana di m}}
Assumo prior uniforme in $\log{m}$: $\prob{(m)}\propto\frac{1}{m}$, \mykeyword{posterior} $\pi(m)\propto L(\mu|m)\prob{(m)}=\frac{6\mu}{m^3}-\frac{6\mu^2}{m^4}$ ($\prob{(H)}=\frac{\prob{(E|H)\prob{(H)}}}{\prob{(E)}}$): la posterior va normalizzata tra $[\mu,+\infty]$ (likelihood nulla  in $[0,\mu]$), $\pi(m)=\frac{3\mu^2}{m^3}-\frac{3\mu^3}{m^4}$, il massimo fornisci una stima bayesiana di m: $\hat{m}=\frac{4}{3}\mu$.
\end{block}
\end{column}\end{columns}
\begin{block}{Limite superiore/bilaterale}
\mykeyword{Limite superiore per m di $U(0,m)$ da mediana campionaria}: $\int_{\mu}^m6\frac{\tilde{\mu}}{m^2}(1-\frac{\tilde{\mu}}{m})\,d\tilde{\mu}=\cl=0.9$, con $y=\frac{\mu}{m}$ si risolve $F(\mu)=y^2(3-2y)=1-\cl$ ha soluzione $y=0.1958$ quindi $m<\frac{\mu}{0.1958}$.
\mykeyword{Limite bilaterali simmetrici per m da mediana campionaria}: $F(\mu)=y^2(3-2y)=(1-\cl)/2$ $y_+=0.13535$, $\int_{\mu}^m6\frac{\tilde{\mu}}{m^2}(1-\frac{\tilde{\mu}}{m})\,d\tilde{\mu}=(1-\cl)/2$ quindi $F(\mu)=y^2(3-2y)=(1+\cl)/2$ e $y_-=0.8647$: Limiti bilaterali con $\cl=0.9$ sono $\frac{\mu}{0.8647}<m<\frac{\mu}{0.13535}$
\end{block}
\begin{block}{Limiti credibilit\'a Bayesiana}
prior uniform, $\cr=90\%$. La posterior \'e $\pi(m)=\frac{2\mu}{m^2}-\frac{2\mu^2}{m^3}$ diversa da zero in $[\mu,\+infty]$ e la cumulante $F(m)=\int_{\mu}^m\pi(\tilde{m})\,d\tilde{m}=1+\mu(\frac{\mu}{m^2}-\frac{2}{m})$:
\begin{align*}
&F(m_{min})=\frac{1-\cl}{2}\\
&F(m_{max})=\frac{1+\cl}{2}\\
&\frac{\mu}{1-\sqrt{(1-\cl)/2}}<m<\frac{\mu}{1-\sqrt{(1+\cl)/2}}
\end{align*}
dove ho scelto le soluzioni maggiori di $\mu$, per il limite superiore si ha $m<66.26$.
\end{block}
\begin{block}{Limiti confidenza al $90\%\cl$ a la FC}
Imponiamo il LR-ordering e coverage $90\%$
\begin{equation*}
\lambda\propto\mu\prob{(\mu;m)}=6\frac{\mu^2}{m^2}(1-\frac{\mu}{m})
\end{equation*}
LR \'e unimodale: un intervallo con $\lambda(m_1)=\lambda(m_2)$, mentre il coverage richiede $\int_{m_1}^{m_2}\prob{(\mu;m)}\,d\mu=\cl$ da cui si ottiene il sistema di equazioni
\begin{align*}
&y_2^3-y_1^3=\cl\\
&y_2^2-y_1^2=\cl\\
&y_1=\frac{m_1}{m},\ y_1=\frac{m_1}{m}\\
&\frac{\mu}{0.96804}<m<\frac{\mu}{0.192606}
\end{align*}
\end{block}
\end{wordonframe}

\begin{wordonframe}{math }
\begin{block}{Soluzione equazione di secondo grado}
\begin{align*}
&4a(ax^2+bx+c)=0\\
&4a^2x^2+4abx+4ac+b^2-b^2=0\\
&(2ax+b)^2=b^2-4ac
\end{align*}
\end{block}

\end{wordonframe}
