%\documentclass[asd-beamer.tex]{subfiles}%! TEX root = asd-beamer.tex
%
%\begin{document}

\section{RegLez 20/21}

\begin{frame}[allowframebreaks]{List of todos}\linkdest{rl20todo}
\listoftodos
\end{frame}

\begin{frame}[allowframebreaks]{List of musts}\linkdest{rl20must}
\listofmusts
\end{frame}

\begin{itemize}
\item 25/09/2020 Presentazione e informazioni pratiche sul corso. Introduzione generale ai contenuti e obiettivi del corso. Concetto di inferenza statistica e sua relazione con le scienze sperimentali. Overview dei tipi di inferenza discussi nel corso.
\item 29/09/2020 Probabilita' frequentista. Formulazione di Von Mises, con esempi. Probabilita' matematica, assiomi di Kolmogorov e relazione con probabilita' frequentista. Probabilita' soggettiva (degree-of-belief). Teoremi elementari sulla probabilita' di Kolmogorov. Definizioni: Probabilita'congiunta, probabilita' condizionata, eventi indipendenti, Legge della Probabilita' Totale. Teorema di Bayes.
\item 30/09/2020  Semplici esempi ed asercizi su probabilit\'a, probabilit\'a condizionata, indipendenza, T. di Bayes. Esempio elementare di rottura di indipendenza (\todo{fallacia di Berkson}). \todo{Efficienza di coppia di contatori}.
\item 02/10/2020 Osservabili e variabili aleatorie. Distribuzioni di probabilita' (pmf) discrete e continue, densita' di probabilita' (pdf), Cumulanti (cdf). \must{Distribuzioni congiunte}, marginalizzazione, indipendenza. Media campionaria e Legge dei Grandi Numeri. Dimostrazione limitata di LLN tramite Disuguaglianza di Chebishev. Valore di aspettazione e sue proprieta'. Varianza e Covarianza. Indipendenza e correlazione.
\item 06/10/2020 Definizione di Momenti di una distribuzione. \todo{Formule generali per ricavare la distribuzione di una Statistica}. Distribuzione della somma aritmetica di due statistiche. Espansione dei momenti di una statistica: formule approssimate di uso comune(''propagazione errori'') e relativi pitfalls. Momenti della distribuzione di Cauchy. Funzione Generatrice dei Momenti e sue proprieta'. Esempi: momenti di d. di Bernoulli e di d. Uniforme. Funzione Caratteristica di una distribuzione e sue proprieta'. Teorema di Paul Levy. Teorema del Limite Centrale, e Distribuzione Gaussian
\item 07/10/2020 Trasformazioni di variabili in 1D, caso di $y=x^2$, problema del spara palline (distribuzione di Cauchy). Trasformazioni di variabili pi\'u dimensioni. Propagazione degli errori, e caso della somma, prodotto e rapporto, con conto esplicito per due variabili indipendenti uniformi. Calcolo delle densit\'a di  probabilit\'a nel caso di somma, prodotto e rapporto di due variabili aleatorie ($z=x+y$,$z=xy$,$z=x/y$), in generale, e per due uniformi ($xy$ e $x/y$ lasciato da verificare per casa). Discussione sulla validit\'a della propagazione degli errori per i rapporto di uniformi.
\item 09/10/2020 Processi di Bernoulli e derivazione frequentista della distribuzione Binomiale, e della sua f. caratteristica. Distribuzione di Poisson e sua Caratteristica. Distribuzione Esponenziale derivata dalla Poisson, e sua caratteristica. Definizione di funzione di Likelihood, e sua proprieta' fondamentali. Esempi likelihood Bernoulli, Poisson, Gaussiana. Combinazione di f. di likelihood di misure indipendenti.
\item 13/10/2020 Concetto di inferenza funzionale. Uso della Likelihood a scopo inferenziale, e sue proprieta' intuitive. Esempio: combinazione di due misure gaussiane indipendenti. Utilizzo del teorema di Bayes: in modalita' frequentista e in modalita' soggettivista. Esempi su test medici e su identificazione di particelle. Probabilita' a priori, a posteriori, rapporto di likelihood. Concetto di prior impropria, ''Betting odds'', Bayes factor, \todo{fattore di reiezione}. Principio di Likelihood - impatto nella pratica della analisi statistica.
\item 14/10/2020 Calcolo della distribuzione del rapporto di due variabili casuali estratte da distribuzioni Gaussiane N(0,1): distribuzione di Cauchy. Propriet\'a della distribuzione di Cauchy e funzione caratteristica. \todo{Calcolo della distribuzione della della somma di due o N variabili estratte da una distribuzione di Cauchy}. \todo{Calcolo della distribuzione della media aritmetica di N variabili estratte da una distribuzione di Cauchy e conseguenze sulla legge dei grandi numeri}. Uso della Likelihood a scopo inferenziale: combinazione di due o pi\'u misure indipendenti estratte da una distribuzione uniforme $U[0,m]$. Esercizio lasciato per casa: calcolare la probabilit\'a $P(max(x,y)=C)$ di due estrazioni x,y da un dado a 6 facce.
\item 20/10/2020 Fattorizzazione della Likelihood e statistiche sufficienti. Esempi Gaussiana e Uniforme. Statistiche sufficienti, e statistiche sufficienti minimali. Teorema di Darmois. Definizione di Score function, Informazione di Fisher e Matrice Informazione. Condizioni per additivita', crescenza e linearita' dell'Informazione di Fisher.
\item 21/10/2020 Breve riassunto della definizione di statistica sufficiente e esempi con la Gaussiana: la media aritmetica come statistica sufficiente per $\mu$ (noto $\sigma^2$), e lo scarto quadratico medio rispetto a $\mu$ come statistica sufficiente per $\sigma^2$ (noto $\mu$), la media aritmetica come statistica sufficiente per il parametro p di una distribuzione binomiale $f(k;n,p)$. Per casa: fare la stessa cosa per $\mu$ della Poisson, e lambda per la distribuzione esponenziale. Applicazione del teorema di Darmois con determinazione delle statistiche sufficienti per la gaussiana (per i differenti casi per $\mu$ e $\sigma^2$), e per la distribuzione binomiale per fare inferenza sul parametro p. Derivata la distribuzione di xmax per N estrazioni da una distribuzione $U[0,1]$, e mostrato che xmax \'e una statistica sufficiente attraverso la definizione. Discusso il fatto che la \must{distribuzione uniforme viola la condizione di Darmois}.
\item 23/10/2020 Fattorizzazione esplicita delle pdf appartenenti alla famiglia esponenziale. \must{Esempio fattorizzazione distribuzione Uniforme. Esempi informazione di Fisher Gaussiana, Esponenziale, Uniforme (non lineare)}. Riduzione della Informazione in mancanza di statistiche sufficienti: esempi accettanza finita (esponenziale troncato), e istogrammazione dei dati.
\item 27/10/2020 Concetto di Stima Puntuale. Come si quotano le stime puntuali nelle pubblicazioni scientifiche. Stimatori: consistenza, bias, varianza. \todo{Disuguaglianza di Cramer-Rao}. Minimum Variance Bound e condizioni sotto le quali si raggiunge. Efficienza di uno Stimatore. \todo{Unicita' della funzione dei parametri ottimalmente stimabile}. Estensione di Cramer-Rao a piu' variabili. \todo{Esempio risoluzione Esponenziale}.
\item 28/10/2020 Informazione di Fisher: Breve introduzione, caso della Bernoulli, Binomiale e Poisson. Disuguaglianza di Cramer-Rao: Breve introduzione, caso della Binomiale, Poisson, Gaussiana con varianza conosciuta, Gaussiana con media conosciuta. Uniforme e considerazioni su applicabilit\'a della disuguaglianza di Cramer-Rao nel caso di uniforme. Per casa: Uniforme con statistica xmin: Studio di consistenza, bias, sufficienza, calcolo dell'Informazione di Fisher, discussione su applicabilit\'a della disuguaglianza di Cramer-Rao, calcolo del valore di aspettazione e varianza, andamenti asintotici della varianza e della PDF.
\item 30/10/2020 Metodi generali per la costruzione di stimatori consistenti. Metodo dei momenti, sue proprieta' e applicazioni. Metodo generale degli stimatori impliciti. Lo stimatore di Massima Likelihood (MLE) e le sue proprieta'. Riduzione del bias: alcune tecniche di uso comune.
\item 03/11/2020 Considerazioni pratiche sul calcolo del MLE, e stima numerica della sua varianza. MLE per istogrammi, schemi di binning, e confronto con la versione ''unbinned''. Extended Likelihood vs. regular Likelihood. \todo{Limiti asintotici del MLE per istogrammi, relazione con i ''fits di chiquadro'', formule semplificate di uso comune e relativi pitfalls}. Stimatore dei Minimi Quadrati (LSE), e sue proprieta'. Caso lineare, B.L.U.E. e Teorema di Gauss-Markov.
\item 04/11/2020 Soluzione Esercizio Xmin del compito del 28/5/2018. Richiami sule propriet\'a del MLE. Calcolo del MLE per il parametro ''p'' della distribuzione binomiale, e calcolo del suo valore di aspettazione e della sua varianza. Lasciato per casa di fare lo stesso esercizio con la distribuzione di Poisson. Calcolo del MLE per $\mu$ e $\sigma^2$ della Gaussiana. Discussione del bias sullo scarto quadratico media e definizione dello stimatore unbiased. Mostrato come l'\todo{MLE e' collegato alla famiglia esponenziale del Teorema di Darmois}, usando come esempio il caso della distribuzione esponenziale (tau VS lambda)
\item 06/11/2020 Outliers. Concetto di Robustezza di uno stimatore. Concetto di incertezza sistematica. Esempi storici. Stimatori di Location parameters. Stimatori da p-norme generalizzate. La mediana: distribuzione e proprieta'. Es. applicazione alla D.di Cauchy. Stimatore robusto di Hodges-Lehmann. Altri esempi: Moda e Mid-range. Applicazione alla D. Uniforme. Stime basate su medie troncate. Approccio minimax alla robustezza delle stime.
\item 10/11/2020 Esercizi riepilogativi sulla stima puntuale. Svolti i punti 1,2,3,4,5,6 del Problema 1 del Compito di esame del 09/11/2018: studio della statistica $x_2$ estratta da una distribuzione $U[0,m]$, dove $x_2$ e' l'estrazione immediatamente minore di $x_{max}$, per la stima del parametro ''m''. Svolto il problema 1 del Compito d'esame del 01/02/2018: applicazione del principio di Likelihood al problema delle risposte multiple (likelihood con un parametro discreto ''x'' e parametro continuo ''m'' o ''$\sigma^2$'') sia in caso di estrazione da una distribuzione uniforme $U[x-m/2,x+m/2]$ con ''m''  ignoto, o da una distribuzione Gaussiana $G(x,\sigma^2)$ con $\sigma^2$ ignota. Da svolgere per casa i punti 1,2,3,4,5 del problema 2 del Compito d'Esame del 09/11/2018.
\item 11/11/2020 Limitazioni del concetto di Stima Puntuale. Esempi. Introduzione alla Stima Intervallare. Principi di Stima Intervallare Bayesiana. Credibilita'. Costruzione di Regioni Credibili. Funzioni di ordinamento. \todo{Uso e vantaggi del Posterior Ordering}. Prior improprie. Esempio Poisson.
\item 17/11/2020 Introduzione ai concetti di Coverage e Confidence Level. Algoritmi di ordinamento. Costruzione di Neyman: bande di confidenza e regioni di confidenza. Esempi svolti: confronto regioni di confidenza frequentiste-Bayesiane per Poisson e Uniforme. Relazione tra Credibilita' media e Coverage media. Es. tabella di limiti Poisson, min/max/centrali. Problema del ''flip-flopping''
\item 18/11/2020 Possibilita' di regioni di confidenza vuote e discussione del significato. Uso del Likelihood-Ratio come funzione di ordinamento, e sue proprieta'. ''Unified approach'' di Feldman-Cousins, e concetto di ''sensitivity''. Definizione di ''pivotal quantity''. \must{Il LR come pivot asintotico (Teorema di Wilks)} e suo uso per la determinazione approssimata di Regioni di Confidenza. Definizione della distribuzione ''chi-quadro''
\item 20/11/2020 Stima intervallare: Esercizio 1 del compito di esame del 28/05/2018 (Dungeons\&Dragons). A lezione punti 3/4: Calcolati i limiti bayesiani e frequentisti del numero di facce del dato, nel caso di estrazione $n=1$. I limiti frequentisti sono stati fatti con l'ordering ''n'' decrescente e con LR ordering (F-C). Esercizio 2 del 9/11/2018 (Onde Gravitazionali) punti 6,7,8,9 : Calcolo dell'intervalli di confidenza con LR ordering nel caso di distribuzione esponenziale con una misura (esercizio delle onde gravitazionali).
\item 24/11/2020 Introduzione al concetto di Test di Ipotesi. Definizioni: Ipotesi Nulla, Ipotesi semplici e composte, regione critica, errori di tipo I e II. Problemi di Classificazione come Test di Ipotesi. Proprieta' dei tests: potenza, consistenza, unbiasedness. Concetto di test MP e UMP. Lemma di Neyman-Pearson: enunciato, dimostrazione, e significato pratico. Test UMP unilaterale per distribuzioni della famiglia esponenziale
\item 25/11/2020 Considerazioni sul testing in situazioni generiche. Indicazioni per tests Locally Most Powerful. Test LMP generale basato sul Fisher Score. Likelihood-Ratio Test: motivazione, uso e proprieta' asintotiche. La distribuzione chi-quadro e alcune sue proprieta'
\item 27/11/2020 Test Neyman-Pearson unilaterale con distribuzioni Gaussiane con medie differenti ($H0:\mu0$ e $H1:\mu1>\mu0$) e stessa varianza $\sigma^2$. Calcolo della soglia $c_{\alpha}$ dato un livello di significativa' alpha, e calcolo del power(mu1) del test. Stesso test ripetuto utilizzando la \todo{statistica di test $N-$ (definita come il numero di estrazioni $x_i$ minori di $\mu0$)}, anche detto Sign Test, e confronto del power ottenuto rispetto al test UMP di Neyman-Pearson. Esercizio (esame 28/5/2018) del test per criticit\'a con crescita esponenziale di flusso di neutroni di una centrale nucleare, basato sugli ultimi N conteggi di flusso. Utilizzo del LMP test. Identificazione della PDF come della famiglia esponenziale e UMP test.
\item 01/12/2020 Introduzione al concetto di ''Goodness of fit''. Tests di GOF e loro caratteristiche. Misure di GOF, definizione di p-value, unbiasedness. Uso di p-values in GOF test. Rischi di interpretazione e critiche al concetto di p-value. \todo{p-values in caso di $H0$ composta}. Formule per la combinazione di p-values indipendenti. GOF di istogrammi con il LR. GOF di un fit con errori gaussiani con il LR, e test del ''chi-quadrato'' di Pearson.
\item 02/12/2020 Il \must{t-test di Student come applicazione del LR}. Distribuzione della statistica t. Problema del GOF per dati non binnati. Test di Kolmogorov-Smirnov, caratteristiche e precauzioni per l'uso. \todo{Test di Mann-Whitney}, e confronto con K-S.
\item 04/12/2020 Esercizio 1 del compito del 14/11/2019 svolto con gli studenti. Svolti i punti 1,2,3,4,5
\item 09/12/2020 Svolto il problema 2 del compito d'esame del 01/02/2018: GOF del dado a 6 facce sia con il Likelihood Ratio che con il test di Pearson. Nel caso del Likelihood Ratio si e' utilizzato sia il prodotto di poissoniane che la multinomiale, e il Teorema di Wilks per l'approssimazione asintotica applicato ad entrambi i casi. 
\item 11/12/2020 \todo{Svolto il compito del 16/07/2019}: GOF con il test K-S, e combinazione di differenti p-values. GOF con il Likelihood ratio. MLE e test di Neyman-Pearson, con il Likelihood Ratio, e test LMP. \todo{Stima della contaminazione con il metodo dei momenti e confronto con MLE}. Limiti superiori frenquetisti e bayesiani sul parametro f.
\item 15/12/2020 Compito di esame del 16/7/2018: Esercizio 1 completo Esercizio 2.1,2.2,2.3,2.4.2.5,2.9 Punti 2.6 e 2.7 dati per casa.
\item  16/12/2020  esercitazione: \todo{Algoritmo ''QUEST''}
\item 18/12/2020 esercitazione: Problema del setting del contatore con fondo. \todo{Generalizzazione alla ''search'' come combinazione di test+stima intervallare}. \todo{Sensitivity region e sua ottimizzazione}. Applicazione al counting experiment.
\end{itemize}

\section{RegLez 19/20}

\begin{frame}[allowframebreaks]{List of todos}\phantomsection\linkdest{rl19todo}
\listoftodos
\end{frame}

\begin{frame}[allowframebreaks]{List of musts}\linkdest{rl19must}
\listofmusts
\end{frame}

\begin{frame}[allowframebreaks]{Reg Lez 19/20}\phantomsection\linkdest{rl19}
\begin{itemize}
\item 27/09/2019 - lezione: Presentazione e informazioni pratiche sul corso. Introduzione generale ai contenuti e obiettivi del corso. Concetto di \must{inferenza statistica} e sua relazione con le scienze sperimentali. Classificazione di \todo{tipi di inferenza}. Concetto di \must{incertezza statistica e di incertezza sistematica}.
\item 01/10/2019 - lezione: \must{Probabilit\'a frequentista}. \must{Formulazione di Von Mises}, con esempi. Probabilit\'a matematica, \must{assiomi di Kolmogorov} e relazione con probabilit\'a frequentista. \must{Probabilit\'a soggettiva} (degree-of-belief), esempi e applicazione inferenziale.
\item 02/10/2019 - lezione: Teoremi elementari sulla probabilit\'a di Kolmogorov. Definizioni: \must{Probabilit\'a congiunta}, \must{probabilit\'a condizionata}, \must{eventi indipendenti}, \must{Legge della Probabilit\'a Totale}. Alcuni semplici esempi/applicazioni. Teorema di Bayes.
\item 04/10/2019 - lezione: Osservabili e variabili aleatorie. Distribuzioni di probabilit\'a (pmf) discrete e continue, densit\'a di probabilit\'a (pdf), Cumulanti (cdf). Distribuzioni congiunte, marginalizzazione, indipendenza. \must{Media campionaria e Legge dei Grandi Numeri}. Dimostrazione limitata di LLN tramite Disuguaglianza di Chebishev. Valore di aspettazione e sue propriet\'a. Varianza e Covarianza.
\item 08/10/2019 - lezione: Definizione di Momenti di una distribuzione. \must{Formule generali per ricavare la distribuzione di una Statistica}. Distribuzione della somma aritmetica di due statistiche. Espansione dei momenti di una statistica: \must{formule approssimate di uso comune(''propagazione errori'') e relativi pitfalls}. Momenti della distribuzione di Cauchy. Funzione Generatrice dei Momenti e sue propriet\'a. Esempi: momenti di d. di Bernoulli e di d. Uniforme. Funzione Caratteristica di una distribuzione e sue propriet\'a. \must{Teorema di Paul Levy}.
\item 09/10/2019 - lezione: \must{Teorema del Limite Centrale}. La Distribuzione Gaussiana. Processi di Bernoulli. Derivazione frequentista della distribuzione Binomiale, e sua funzione Generatrice.
\item 11/10/2019 - lezione: Distribuzione di Poisson e sua Generatrice. \todo{Distribuzione Esponenziale (derivata dalla Poisson)}, e sua generatrice. Definizione di Likelihood e sua propriet\'a fondamentali. Esempi Likelihood Bernoulli e Uniforme. Principio di Likelihood. Uso della Likelihood a scopo inferenziale. Uso del teorema di Bayes in modo frequentista e in modo soggettivista. Probabilit\'a a priori, a posteriori, rapporto di likelihood, ''betting odds''. \must{Statistiche sufficient e Teorema di Darmois}.
\item 15/10/2019 - esercitazione: Propagazione degli errori, e caso della somma, prodotto e rapporto, con conto esplicito per due variabili indipendenti uniformi. Trasformazioni di variabili in 1D, caso di $y=x^2$, \todo{problema del spara palline (distribuzione di Cauchy)}. Trasformazioni di variabili pi\'u dimensioni, calcolo delle densit\'a di probabilità nel caso di somma, prodotto e rapporto di due variabili aleatorie ($z=x+y,z=xy,z=x/y$), in generale, e per due uniformi. Discussione sulla validità della propagazione degli errori per i rapporto di uniformi, e calcolo di $P(z)$ con ,$z=x/y$ rapporto di due Gaussiane. Proprietà della distribuzione di Cauchy, introduzione alla somma di due variabili di Cauchy e conseguenze per la media di Cauchy e sul teorema dei grandi numeri.
\item 18/10/2019 - esercitazione: Esercizi su inferenza bayesiana: \todo{esercizio sul fascio di particelle (exe 1.3 dell'eserciziario del Cowan)}, esercizio sulla qualit\'a delle casse di munizioni, punto 1.1 e 1.2 del Compito d’Esame 28/05/2018 (Dungeons\&Dragons). Breve riassunto della definizione di statistica sufficiente e esempi con la Gaussiana: la media aritmetica come statistica sufficiente per mu, nota $\sigma^2$, e lo scarto quadratico medio da mu come statistica sufficiente per $\sigma^2$ con $\mu$ noto. Applicazione del teorema di Darmois con determinazione della media aritmetica come statistica sufficiente per la media della Gaussiana e il parametro p della distribuzione binomiale.
\item 22/10/2019 - lezione: \todo{Dimostrazione Teorema di Fattorizzazione}. Esempio fattorizzazione D. Uniforme. Statistiche sufficienti minimali. Informazione di Fisher. Matrice Informazione. Score function. \must{Additivit\'a, crescenza e linearit\'a dell'Informazione di Fisher}. Esempio non-linearit\'a Uniforme. Informazione da statistiche sufficienti. Riduzione della Informazione in mancanza di statistiche sufficienti: esempi accettanza finita (esponenziale troncato), istogrammazione dei dati.
\item 23/10/2019 - esercitazione: Applicazione del teorema di Darmois con forma esponenziale multidimensionale: statistiche sufficienti jointly per $\mu$ e $\sigma^2$ della gaussiana. Densit\'a di probabilità di $x_{max}$ per N estrazioni da distribuzione uniforme $U[0,m]$. Dimostrazione che $x_{max}$ \'e una statistica sufficiente per m, utilizzando il teorema di fattorizzazione. Calcolo dell'informazione di Fischer per il parametro $\mu$ della distribuzione di Poisson direttamente dalle estrazioni $k_i$ e con la statistica sufficiente $\bar{k}$.
\item 25/10/2019 - lezione: Concetto di Stima Puntuale. Come si quotano le stime puntuali nelle pubblicazioni scientifiche. Stimatori: consistenza, bias, varianza. Disuguaglianza di Cramer-Rao. \must{Minimum Variance Bound e condizioni sotto le quali si raggiunge}. Efficienza di uno Stimatore. \todo{Unicit\'a della funzione dei parametri ottimalmente stimabile}.
\item 29/10/2019 - lezione: Esempi applicazione Cramer-Rao a distribuzioni Esponenziale, Normale, Uniforme. Cenni a metodi di riduzione del bias. Disuguaglianza di Cramer-Rao in pi\'u variabili. \must{Metodi generali per la costruzione di stimatori consistenti}. Metodo dei momenti; sue proprieta' e applicazioni. Metodo generale degli stimatori impliciti. Stimatore di Massima Likelihood (MLE) e sue propriet\'a.
\item 30/10/2019 - esercitazione: \todo{Algoritmo QUEST} (\url{https://www.researchgate.net/publication/16355122_QUEST_A_Bayesian_adaptive_psychometric_method}): likelihood , Informazione di Fisher, metodo iterativo basato su approccio bayesiano, Accenni sullo stimatore e sulla terminazione dell'algoritmo in approccio frequentista, Accenni sull'efficienza dello stimatore.
\item 05/11/2019 - lezione: Considerazioni pratiche sul calcolo del MLE, e stima numerica della sua varianza. MLE per istogrammi, schemi di binning, e confronto con la versione ''unbinned''. Extended Likelihood vs. regular Likelihood. \todo{Limiti asintotici del MLE per istogrammi}, relazione con i ''fits di chiquadro'', formule semplificate di uso comune e relativi pitfalls. Stimatore dei Minimi Quadrati (LSE), e sue propriet\'a. \todo{Caso lineare, B.L.U.E.} e \must{Teorema di Gauss-Markov}.
\item 08/11/2019 - esercitazione: Richiami sule proprietà del MLE. Calcolo del MLE per il parametro ''p'' della distribuzione binomiale, e calcolo del suo valore di aspettazione e della sua varianza. Calcolo dell'informazione di Fischer rispetto al parametro p e mostrato l'MLE e' uno stimatore efficiente per questo caso. Lasciato per casa di fare lo stesso esercizio con la distribuzione di Poisson. Calcolo del MLE per $\mu$ e $\sigma^2$ della Gaussiana. Calcolo del bias sullo scarto quadratico media e calcolo della varianza degli estimatori biased a unbiased. Mostrato come l'MLE \'e collegato alle statistiche sufficienti joint, precedentemente calcolate con il Teorema di Darmois.
\item 08/11/2019 - lezione: Outliers. Concetto di Robustezza di uno stimatore. Stimatori di Location parameters. Stimatori da p-norme generalizzate. \todo{Stime di Moda, Mid-range, Mediana}. Distribuzione e Propriet\'a della Mediana.
\item 12/11/2019 - lezione: \todo{Applicazione del metodo minimax} alle stime basate su medie troncate. \todo{Stimatore robusto di Hodges-Lehmann}. Limitazioni del concetto di Stima Puntuale. Esempi. Introduzione alla \must{Stima Intervallare}. Principi di Stima Intervallare Bayesiana. Credibilit\'a. Costruzione di Regioni Credibili. Funzioni di ordinamento. Uso e vantaggi del Posterior Ordering. Prior improprie. Esempio Poisson. Introduzione ai concetti di Coverage e Confidence Level.
\item 15/11/2019 - lezione: \must{Costruzione di Neyman}: bande di confidenza e regioni di confidenza. \todo{Relazione tra Credibilit\'a media e Coverage media}. Algoritmi di ordinamento, e propriet\'a del P-Ordering. Esempi svolti: confronto regioni di confidenza frequentiste-Bayesiane per Poisson e Uniforme. \must{Problema del ''flip-flopping''}. Possibilit\'a di regioni di confidenza vuote e discussione del significato.
\item 19/11/2019 - lezione: Uso del Likelihood-Ratio come funzione di ordinamento e sue proprieta'. \must{''Unified approach'' di Feldman-Cousins}. Concetto di ''pivotal quantity''. \must{Il LR come pivot asintotico (Teorema di Wilks)} e suo uso per la determinazione approssimata di Regioni di Confidenza. Definizione e propriet\'a delle distribuzioni ''chi-quadro''. Introduzione al concetto di \must{Test di Ipotesi}. Definizioni: Ipotesi Nulla, Ipotesi semplici e composte, regione critica, errori di tipo I e II. Problemi di Classificazione come Test di Ipotesi. Propriet\'a dei tests: potenza, consistenza, unbiasedness.
\item 20/11/2019 - esercitazione: Intervalli di confidenza frequentisti sul parametro m della distribuzione $U[0,m]$ con singola misura x e con N misure usando come statistica $x_{max}$ ($x_{min}$ proposto per casa), con LR-Ordering alla F.C. Calcolo della distribuzione del $\lambda=-2log(LR)$ e calcolo del relativo intervallo di confidenza. Calcolo dell'intervalli di confidenza con LR ordering nel caso di distribuzione esponenziale con una misura (esercizio delle onde gravitazionali).
\item 22/11/2019 - esercitazione: Svolti interamente esercizi 1 e 2 del \must{compito di esame del 28/05/2018} (Dungeons\&Dragons). Calcolati i limiti bayesiani e frequentisti del numero di facce del dato, nel caso di estrazione $n=1$. I limiti frequentisti sono stati fatti con l'ordering ''n'' decrescente e con LR ordering (F-C). Calcolata la distribuzione di $x_{min}$, $E[x_{min}]$ e $Var[x_{min}]$. Calcolato l’estimatore $\hat{m}$ sia con il metodo dei momenti che MLE, e calcolati valori medi e varianze dei due estimatori (con il limite asintotico). Lasciato per casa di calcolare l’informazione di Fischer e verificare la diseguaglianza di Cramer-Rao. Calcolata la distribuzione di $\hat{m}$ e il suo limite asintotico (esponenziale). Calcolati i limiti di confidenza frequentisti con $x_{min}$ crescente (lower limit) probability ordering (upper limit), e LR ordering (impostato il sistema, non si risolve analiticamente).
\item 26/11/2019 - esercitazione: Svolto interamente l’esercizio 1) del compito di esame del 01/02/2018 (studente che deve scegliere la risposta corretta in una prova scritta a risposta multipla). Svolti interamente i punti 8),10),11),12),13),14), 17) e 18) del compito d’esame del 16/07/2019 (Generatori pseudo-casuali dei moderni computer).
\item 29/11/2019 - lezione: Concetto di test MP e UMP. \must{Lemma di Neyman-Pearson}: enunciato, dimostrazione e suo significato pratico. Test UMP unilaterale per distribuzioni della famiglia esponenziale. Motivazioni per \must{tests Locally Most Powerful}. Test LMP generale basato sul Fisher Score. \must{Likelihood-Ratio Test}: motivazione, uso e propriet\'a asintotiche.
\item 03/12/2019 - esercitazione: Test di ipotesi: Test N-P su gaussiane con medie differenti. Sign test su gaussiane con medie differenti. Esercizio (esame 28/5/2018) del test per criticit\'a con crescita esponenziale di flusso di neutroni di una centrale nucleare, basato sugli ultimi N conteggi di flusso. Utilizzo del LMP test. Identificazione della PDF come della famiglia esponenziale e UMP test. Confronto delle due statistiche.
\item 06/12/2019 - esercitazione: Test di ipotesi: Continuazione dell esercizio 3 dell'esame del 28/5/2018. Punti 2 e 3 (soglia per il test, e power). Esercizio 1 dell'esame del 14/01/2019. Punti 1,2,3,4,5a (no GOF) svolti in classe dagli studenti con supervisione dell'esercitatore.
\item 10/12/2019 - lezione: Introduzione al concetto di ''Goodness of fit''. \must{Tests di GOF e loro caratteristiche}. Misure di GOF, definizione di p-value, unbiasedness. Uso di p-values in GOF test. Rischi di interpretazione e critiche al concetto di p-value. p-values per il caso di H0 composta. Formule per la combinazione di 2, o di N p-values.
\item 13/12/2019 - lezione: GOF per istogrammi con il LR. GOF di un fit con errori gaussiani con il LR, e test del ''chi-quadrato'' di Pearson. \todo{Problema del GOF per dati non binnati}. Test di Kolmogorov-Smirnov, caratteristiche e istruzioni per l'uso. Il \must{t-test di Student come applicazione del LR}. Distribuzione della statistica t.
\item 17/12/2019 - esercitazione: \todo{Problema 2 del compito di esame del 01/02/2018}: GOF con istogramma a 6 bin (facce del dado) della distribuzione uniforme con likelihood ratio sia con prodotto di poissoniane che con distribuzione multinomale, e con il test di Pearson in approssimazione asintotica. Domande 1,2,3,4,5,6,7,9,15, e 16 del \todo{compito di esame del 16/07/2019}: GOF con K-S della distribuzione uniforme, GOF con combinazione di p-values; GOF di istogramma come sopra; \todo{derivazione della distribuzione triangolare dalla differenza di due estrazioni da distribuzione $U[0,1]$}; test di Neyman Pearson, Likelihood ratio, e LMP con distribuzione binomiale.
\item 18/12/2019 - lezione: \todo{Concetto di ''search'' come combinazione di test+stima intervallare}. Sensitivity region e sua ottimizzazione. Applicazione al counting experiment.
\end{itemize}
\end{frame}

\section{RegLez 18/19}

\begin{frame}[allowframebreaks]{List of todos}\phantomsection\linkdest{rl18todo}
\listoftodos
\end{frame}

\begin{frame}[allowframebreaks]{List of musts}\phantomsection\linkdest{rl18must}
\listofmusts
\end{frame}


\begin{frame}[allowframebreaks]{Reg Lez 18/19}\phantomsection\linkdest{rl18}
\cite{reg18}.
%\listofkeywords
\begin{itemize}
\item 18/09/2018 - Concetto di \must{inferenza statistica} e sua relazione con le scienze sperimentali. Classificazione di \todo{tipi di inferenza}. Concetto di \must{incertezza statistica e sistematica}.
\item 19/09/2018 - \must{Probabilit\'a frequentista} (von Mises). \must{Probabilit\'a soggettiva}. \must{Probabilit\'a matematica}, sigma-algebre, assiomi di Kolmogorov. Teoremi elementari di probabilit\'a. Definizioni: \must{Probabilit\'a congiunta}, \must{probabilit\'a condizionata}, eventi indipendenti, \must{Teorema di Bayes}, \must{Legge della Probabilit\'a Totale}.
\item 21/09/2018 (Esercizi) Probabilit\'a, probabilit\'a condizionata, assiomi di Kolmogorov. Esempi specifici su $P(A \cap B)$ e eventi indipendenti, concetto di indipendenza nel caso di pi\'u di due classi di eventi. Esempi con dadi al casino e indipendenza con mazzo da 52 e 51 carte.
\item 25/09/2018 - \must{Osservabili e variabili aleatorie}. Distribuzioni di probabilit\'a (pmf) discrete e continue, densit\'a di probabilit\'a (pdf), Cumulanti (cdf). \must{Distribuzioni congiunte, marginalizzazione, indipendenza}. \must{Media campionaria e Legge dei Grandi Numeri}. Dimostrazione limitata di LLN tramite Disuguaglianza di Chebishev. Valore di aspettazione e sue propriet\'a. Varianza e Covarianza.
\item 26/09/2018 - Esempio di uso del teorema di Bayes (esempio dei dadi); Propriet\'a dell'operatore di valore di aspettazione; $\var{}=E(x^2)-E(x)^2$; Calcolo della varianza di somma di due variabili aleatorie e introduzione della covarianza e della correlazione; Correlazione vs indipendenza (caso $y=x^2$), Uso della distribuzione cumulante ed esempio con $prob(max(x,y)=C)$ per x e y due lanci di dadi. Cambiamento di variabile in una o pi\'u dimensioni. Generazione di una esponenziale a partire da una uniforme, $y=x^2$, problema della lanciapalle da tennis (distr Cauchy).
\item 28/09/2018 - Espansione dei \must{momenti di una statistica}:formule approssimate di uso comune(\must{propagazione errori}) e relativi pitfalls. Momenti di una Distribuzione. Momenti della distribuzione di Cauchy. \must{Funzione Generatrice dei Momenti} e sue propriet\'a. Esempio Uniforme. \must{Funzione Caratteristica}, e sue propriet\'a. \must{Teorema di Paul Levy}. \must{Teorema del Limite Centrale}. Distribuzioni Gaussiane.
\phantomsection\linkdest{ottobre}
\item 02/10/2018 - \must{Processi di Bernoulli}. Derivazione frequentista della \must{distribuzione Binomiale}, e sua funzione Generatrice. \must{Distribuzione di Poisson} e sua Generatrice. \must{Distribuzione Esponenziale }(derivata dalla Poisson), e sua generatrice. Definizione di \must{Likelihood} e sua propriet\'a di base. \must{Principio di Likelihood}.
\item 03/10/2018 - \must{Propagazione degli errori}, e caso della somma, prodotto e rapporto, con conto esplicito per due variabili indipendenti uniformi. Calcolo delle densit\'a di probabilit\'a nel caso di \must{somma, prodotto e rapporto di due variabili aleatorie} ($z=x+y$,$z=xy$,$z=x/y$), in generale, e per due uniformi. Discussione sulla validit\'a della propagazione degli errori per i rapporto di uniformi, e calcolo di $P(z)$ con ,$z=x/y$ rapporto di due Gaussiane. Propriet\'a della \must{distribuzione di Cauchy} (momenti, somma di due variabili di Cauchy e conseguenze per la media di Cauchy e sul teorema dei grandi numeri).
\item 05/10/2018 - Derivazione della f. Caratteristica della Cauchy. Media di variabili Cauchy, e confronto con caso Gaussiano. Discussione del problema del processo esponenziale, importanza della scelta dell'Ensemble; \todo{paradosso del Blue Bus: calcolo della correlazione}. Relazione con bias di selezione e trigger. Cenni alla Fallacia di Berkson, con esempi.
\item 09/10/2018 - Uso della Likelihood a scopo inferenziale. Uso del teorema di Bayes in modo frequentista e in modo soggettivista. Probabilit\'a a priori, a posteriori, rapporto di likelihood, "betting odds". \must{Statistiche sufficienti}. \must{Teorema di Darmois}.
\item 10/10/2018 - \must{Esercizi su inferenza bayesiana}: esercizio sul fascio di particelle (exe 1.3 dell’eserciziario del Cowan), esercizio sulla qualit\'a delle casse di munizioni, punto 1.1 e 1.2 del \must{Compito d’Esame 28/05/2018} (Dungeons$\&$Dragons). Breve riassunto della definizione di statistica sufficiente e esempi con la Gaussiana: la media aritmetica come statistica sufficiente per $\mu$, nota $\sigma^2$, e lo scarto quadratico medio da $\mu$ come statistica sufficiente per $\sigma^2$ con $\mu$ noto. Determinazione della \must{distribuzione di $x_{max}$} per la distribuzione uniforme $U[0,m]$ e dimostrazione tramite la definizione di statistica sufficiente che \must{$x_{max}$ \'e una statistica sufficiente per il parametro m}.
\item 16/10/2018 - \must{Informazione di Fisher}. \todo{Matrice Informazione}. Score function. Additivit\'a, crescenza e linearit\'a dell'Informazione di Fisher. Esempi. Informazione da statistiche sufficienti. \todo{Riduzione della Informazione in mancanza di statistiche sufficienti}: esempi accettanza finita (esponenziale troncato), istogrammazione dei dati.
\item 17/10/2018 - Concetto di \must{Stima Puntuale}. \must{Stimatori}: consistenza, bias, varianza. \must{Disuguaglianza di Cramer-Rao}, Minimum Variance Bound e condizioni sotto le quali si raggiunge. Efficienza di uno Stimatore. Esempi svolti: esponenziale, media della gaussiana. Esempio $I_F$ non lineare: stima di upper limit di distribuzione uniforme usando media aritmetica vs $x_{max}$.
\item 19/10/2018 - Breve riassunto e \must{applicazioni del Teorema di Darmois} con estrazione della relativa statistica sufficiente: Gaussiana e Binomiale. Calcolo dell'Informazione di Fischer per Poisson e Binomiale, rispettivamente per la media $\mu$ e per la probabilit\'a p. Quando Cramer-Rao non vale: verifica del fatto che la disuguaglianza di Cramer-Rao per $x_{max}$ da N-estrazioni $x_i$ da una distribuzione uniforme non vale. Studio della statistica $x_{min}$, come da problema 2 del compito di esame del 25/05/2018: calcolo della distribuzione di $x_{min}$ calcolo del valore di aspettazione, della varianza, nel limite finito e nel limite asintotico. Studio delle proprietà (consistenza e bias) di $x_{min}$ come estimatore di m (dove $\hat{m} = N \cdot x_{min}$). Dimostrazione che la statistica $x_{min}$ non \'e sufficiente per stimare m. Calcolo della informazione di Fischer di $x_{min}$ e verifica della validità o non validità della disuguaglianza di Cramer-Rao.

\item 23/10/2018 - Metodi generali per la \must{costruzione di stimatori consistenti}. \must{Metodo dei momenti}, sue propriet\'a e applicazioni. \must{Metodo degli stimatori impliciti}. \must{Stimatore di Massima Likelihood} (MLE) e sue propriet\'a. Generalit\'a sul bias e metodi per la sua riduzione.
\item 24/10/2018 - Considerazioni pratiche sul \todo{calcolo del MLE}. \todo{MLE per istogrammi}, schemi di binning, e confronto con la versione "unbinned". Stima numerica della varianza del MLE. Extended Likelihood vs. regular Likelihood. \todo{Limiti ad alta statistica e formule semplificate di uso comune}. Stima puntuale con il \must{Metodo dei Minimi Quadrati} e sue proprieta'. Caso lineare e \todo{Teorema di Gauss-Markov}.
\item 26/10/2018 - Ripasso delle \must{propriet\'a del MLE nel caso asintotico} e nel caso finito per \must{distribuzioni della famiglia esponenziale}. MLE per la binomiale $\hat{p}$, e poissoniana $\hat{\mu}$ e studio delle loro proprietà. MLE della Gaussiana $\hat{mu}$ e $\hat{\sigma^2}$ e studio delle loro propriet\'a. MLE per la vita media di una distribuzione esponenziale. Sia nel caso $p(t,\tau)=1/\tau exp(-t/\tau)$ che nel caso $p(t,\lambda)=\lambda exp(-\lambda t)$ e studio delle loro propriet\'a. MLE per $x_{min}$ da distribuzione uniforme: valore di aspettazione, bias, varianza e verifica della non consistenza dell'estimatore. Per casa: \todo{studio del MLE per $x_{max}$}.
\item 30/10/2018 - \todo{Outliers}. Concetto di Robustezza di uno stimatore. Stimatori di Location parameters. Stimatori da p-norme generalizzate. Stime di \must{Moda}, \must{Mid-range}, \must{Mediana}. Distribuzione e Propriet\'a della Mediana. Esempio mediana di distr. Cauchy.
\item 31/10/2018 - \must{Mediana}: Massima LH per uniforme e comportamento asintotico, Esercizio dell'esame a risposta multipla (risposta random, e mediana), distribuzione di probabilit\'a per la mediana nel caso di pdf continua, limite asintotico per la mediana, calcolo della varianza e limite per grandi N; mediana per distribuzione di Cauchy e discussione del MLE e dell'efficienza della mediana. Stimatori $L_{\alpha}$ e efficienze asintotiche.
\phantomsection\linkdest{novembre}
\item 06/11/2018 - Usi della Stima Puntuale e sue limitazioni. Esempi. Introduzione alla \must{Stima Intervallare}. Usi della Stima Intervallare. Principi di {Stima Intervallare Bayesiana}. \must{Credibilit\'a}. Costruzione di \must{Regioni Credibili}. \must{Funzioni di ordinamento}. Uso e vantaggi del \must{Posterior Ordering}. Esempi: Poisson, Uniforme. Introduzione ai concetti di \must{Coverage e Confidence Level}.
\item 07/11/2018 -\todo{QUEST: bayesian adaptive algorithm}. Algoritmo QUEST (\url{https://www.researchgate.net/publication/16355122_QUEST_A_Bayesian_adaptive_psychometric_method}): likelihood , Informazione di Fisher, metodo iterativo basato su approccio bayesiano, stimatore e terminazione dell'algoritmo in approccio frequentista, efficienza dello stimatore.
\item 09/11/2018 - \todo{Relazione tra Credibilit\'a media e Coverage media}. \must{Costruzione di Neyman}, \must{bande e regioni di confidenza}. Algoritmi di ordinamento, e \todo{propriet\'a del Probability Ordering}. \todo{Problema del ''flip-flopping''}. Possibilit\'a di \todo{regioni di confidenza vuote} e discussione del suo significato. Uso del \todo{Likelihood-Ratio come funzione di ordinamento e sue propriet\'a}. \must{"Unified approach" di Feldman-Cousins}. Concetto di \must{"pivotal quantity"}. Il \must{LR come pivot asintotico (Teorema di Wilks)} e suo uso per la determinazione approssimata di Regioni di Confidenza.
\item 13/11/2018 - Richiamo sugli intervalli bayesiani e frequentisti, e regole di ordinamento comune. - Upper limit bayesiano e frequentista per una poissoniana con 0 conteggi. - Costruzione dell'upper limit, lower limit e intervalli di confidenza con ordinamento di probabilit\'a e LR per la poissoniana. - Esercizio dei dadi di $D\&D$ dell'esame di maggio 2018.
\item 14/11/2018 - Introduzione al concetto di \must{Test di Ipotesi}. Definizioni: Ipotesi Nulla, Ipotesi semplici e composte, regione critica, \must{errori di tipo I e II}. Problemi di Classificazione come Test di Ipotesi. \must{Propriet\'a dei tests: potenza, consistenza, unbiasedness, MP, UMP}. \must{Lemma di Neyman-Pearson}, con dimostrazione.
\item 16/11/2018 - \must{Test UMP unilaterale per distribuzioni della famiglia esponenziale}. Motivazioni per \must{tests Locally Most Powerful}. \must{Test LMP unilaterale} di applicabilit\'a generale. \must{Likelihood-Ratio Test}: motivazione, uso e propriet\'a asintotiche. \must{Distribuzioni ''chi-quadro''}: definizioni e propriet\'a.
\item 21/11/2018 - \todo{Intervalli di confidenza frequentisti sul parametro m della distribuzione $U[0,m]$ con singola misura x}: limite superiore e inferiore con ordinamento crescente e decrescente su x, e estrazione degli intervalli di confidenza two-sided simmetrici e tramite l’algoritmo LR-ordering alla F-C. \todo{Intervalli di confidenza frequentisti sul parametro m della distribuzione Uniforme[0,m] con N estrazioni dell’osservabile x} tramite la statistica sufficiente $x_{max}$: limite superiore con ordinamento crescente su $x_{max}$; \todo{intervalli two-sided tramite l’algoritmo di LR-ordering alla F-C}; \todo{calcolo della distribuzione del $\lambda=-2\log{LR}$ e calcolo del relativo intervallo di confidenza}.
\item 23/11/2018 -\must{GOF}. Introduzione al concetto di "Goodness of fit". Tests di GOF e loro caratteristiche. Misure di GOF, definizione di \must{p-value}, unbiasedness. Uso di p-values in GOF test. Rischi di interpretazione, e critiche al concetto di p-value. Formule per la \must{combinazione di 2, o di N p-values}.  
\item 27/11/2018 (M) - \todo{GOF per istogrammi con il LR}. \todo{GOF di un fit con errori gaussiani con il LR}, e \must{test del "chi-quadrato" di Pearson}. Problema del GOF per dati non binnati. \must{Test di Kolmogorov-Smirnov}.
\item 28/11/2018 (M) - Confronto degli intervalli di confidenza frequentisti e bayesiani sul parametro m della distribuzione $U[0,m]$ con N estrazioni, attraverso la statistica sufficiente $x_{max}$. Calcolo della funzione di Coverage e della funzione di Credibilità per entrami i casi e discussione sulla non validità del teorema dei valori medi $E[Cr(xmax)] = E[C(m)]$. Esempio di un test di Neyman-Pearson con ipotesi semplici, nel caso di due distribuzioni gaussiane con stessa varianza $\sigma^2$ ma differenti medie $\mu_0$ e $\mu_1$.
\phantomsection\linkdest{dicembre}
\item  05/12/2018 Test di ipotesi: (esame 28/5/2018) Esercizio del test per criticit\'a con crescita esponenziale di flusso di neutroni di una centrale nucleare, basato sugli ultimi N conteggi di flusso. Utilizzo del MLP test. Identificazione della PDF come della famiglia esponenziale e UMP test. Confronto delle due statistiche. Calcolo della significatività e del power in approssimazione asintotica per N grandi. Calcolo del MLE per piccoli valori della costante di crescita esponenziale. Introduzione del \must{Sign Test}. Calcolo della significativit\'a e del power. Confronto del power con il test UMP su ipotesi semplici con due gaussiane con differente media.
\item Mer 12/12/2018 - \todo{Concetto di "search"} come combinazione di test+stima intervallare. \todo{Sensitivity region} e sua ottimizzazione. Applicazione al counting experiment.
   \end{itemize}
\end{frame}

\section{RegLez 17/18}

\begin{frame}[allowframebreaks]{Reg Lez 17/18}
%\begin{verbatim}
\begin{itemize}
\item lezione: Introduzione generale al corso. Elementi di Probabilita'. Probabilita' frequentista. Probabilit\'a soggettiva. Assiomi di Kolgomorov. Definizioni e identita' di base.
\item esercitazione: Esercizi su: probabilità, probabilit\'a condizionata, assiomi di Kolmogorov. Esempi specifici su $P(A \cap B)$ e eventi indipendenti.
\item esercitazione: Esercizi su: dadi al casino, esempi pratici si $P(A \cap B)$, concetto di indipendenza nel caso di più di due classi di eventi. Esempi di marginalizzazione di una funzione di distribuzione.
\item lezione: Definizioni e elementi base della Statistica. Statistiche, Valore di aspettazione, Osservabili, Indipendenza, Varianza e Covarianza. Distribuzioni di probabilita' discrete e continue, densita' di probabilita' (pdf), Cumulanti (cdf). Trasformazioni delle distribuzioni per cambiamento di variabile. Media campionaria e Legge dei Grandi Numeri.
\item esercitazione: Correlazione vs indipendenza (caso $y=x^2$), uso della distribuzione cumulante. Cambiamento di variabile in una o più dimensioni. Calcolo delle densità di probabilità nel caso di somma, prodotto e rapporto di due variabili aleatorie ($z=x+y,z=xy,z=x/y$).
\item lezione: Momenti di una Distribuzione. Approssimazione dei momenti di una statistica ("propagazione errori"). Funzione Generatrice dei Momenti. Esempi. Processi di Bernoulli. Distribuzione Binomiale. Distribuione di Poisson e distribuzione (Esponenziale) della distanza tra i suoi eventi. Funzione Caratteristica.
\item esercitazione: Introduzione alla binominale e alla poissoniana, calcolo delle funzioni generatrici dei momenti, delle medie e delle varianze - Plot e tabelle di probabilità delle distribuzioni per alcuni casi specifici - Esempio (Binomiale): Calcolo di media e varianza per la variabile di asimmetria $A=U-D/U+D$ e per $eff=k/N$ - Esempio (Poisson) uso della poissoniana e calcolo della PDF per i conteggi in un contatore inefficiente. - Esercizio per casa: Calcolare la PDF per A e per $eff=k/N$ e negative binomial - Introduzione alla distribuzione esponenziale e Cauchy, per exp. calcolo della funzione caratteristica , della media e della varianza. - Esempio: Prescale stocastico e deterministico e fun. caratteristiche - Esercizio per casa: calcolo della fun. caratteristica per Cauchy
\item lezione: Teorema del Limite Centrale e densita' di probabilita' Gaussiana. Definizione e proprieta' generali della Funzione di Likelihood. Esempi. Uso della Likelihood in inferenze di tipo Bayesiano: prior, posterior, betting odds, belief-updating ratio. Cenni su effetti di incertezza sistematica.
\item esercitazione: esempi di inferenza bayesiana: problema della meningite, problema del fascio di particelle, problema delle casse di munizioni, problema del numero del taxi e della temperatura. Esempi e criticità dell'uso di una prior "impropria", come il caso della distribuzione di poisson con conteggio nullo.
\item lezione: Concetto di "riduzione dei dati". Statistiche sufficienti, e statistiche sufficienti minimali. Esempi Uniforme, Esponenziale, Gaussiano. Teorema di Darmois.
\item lezione: Esempio e dimostrazione sufficienza per la statistica $max(x)$ della $U(0,m)$. Definizione di Matrice di Informazione di Fisher. Additivita', crescenza e linearita' dell'Informazione di Fisher. Esempi. Informazione da statistiche sufficienti. Perdita di Informazione in mancanza di statistiche sufficienti: esempi istogrammazione dei dati, accettanza finita (esponenziale troncato).
\item esercitazione: - Informazione di Fisher per esponenziali troncate e informazioni sulle code dell’esponenziale; - Informazione e statistiche sufficienti (il caso della distribuzione uniforme); - Statistiche sufficienti (gaussiana) - Teorema di Darmois (gaussiana, poisson) - Informazione di Fisher (gaussiana, poisson e risultati per binomiale)
\item lezione: Concetto di Stima Puntuale. Stimatori: consistenza, bias, varianza. Disuguaglianza di Cramer-Rao. Minimum Variance Bound e condizioni sotto le quali si raggiunge. Efficienza di uno Stimatore.
\item lezione: Metodi generali per la costruzione di stimatori consistenti. Metodo dei momenti e sue applicazioni. Stimatore di Massima Likelihood (MLE) e sue propriet\'a. Esempio stima simultanea di media e varianza di Gaussiana. Metodi di correzione del bias.
\item esercitazione: Stimatore di massima likelihood e sue proprietà (consistenza, calcolo del valore atteso e della varianza, bias e limiti asintotici) per misure di variabili aleatorie per distribuzioni binomiale, poissoniana, esponenziale, uniforme. Media pesata e sua varianza.
\item lezione: MLE per istogrammi. Extended Likelihood vs. regular Likelihood. Casi particolari. Limiti ad alta statistica e formule semplificate di uso comune. Metodo dei Minimi Quadrati. Consistenza e unbiasedness. Caso lineare e Teorema di Gauss-Markov.
\item lezione: Outliers. Concetto di Robustezza di uno stimatore. Stimatori di Location parameters. Stimatori da p-norme generalizzate. Stime di Moda, Mid-range, Mediana. Distribuzione e Proprieta' della Mediana. Esempio mediana di distr. Cauchy.
\item lezione: Usi della Stima Puntuale e sue limitazioni. Esempi. Usi della Stima Intervallare. Introduzione alla Stima Intervallare. Principi di Stima Intervallare Bayesiana. Credibilita'. Costruzione di Regioni Credibili. Funzioni di ordinamento. Uso e vantaggi del Posterior Ordering. Esempi: Poisson, Uniforme.
\item lezione: Introduzione ai concetti di Coverage e Confidence Level. Proprieta', similarita' e differenze con il concetto di Credibility. Costruzione di Neyman, bande di confidenza. Algoritmi di ordinamento, e proprieta' del Probability Ordering. Esempi: Poisson upper limits, Uniforme (N=1,2); confronto con corrispondenti risultati Bayesiani. Relazione tra Credibilita' media e Coverage media.
\item  lezione: Regioni di Confidenza: problema del "flip-flopping". Possibilit\'a di regioni di confidenza vuote e discussione del suo significato. Uso del LIkelihood-Ratio come funzione di ordinamento e sue proprieta'. "Unified approach" di Feldman-Cousins. Concetto di "pivotal quantity". Il LR come pivot asintotico (Teorema di Wilks) e suo uso per la determinazione approssimata di Regioni di Confidenza. La distribuzione del Chi-quadro e sua interpretazione. 
\item  lezione: Introduzione al concetto di Test di Ipotesi. Definizioni: Ipotesi Nulla, Ipotesi semplici e composte, regione critica, errori di tipo I e II. Problemi di Classificazione visti come Test di Ipotesi. Proprieta' dei tests: potenza, consistenza, unbiasedness, MP, UMP. Lemma di Neyman-Pearson, con dimostrazione. Esistenza di test UMP unilaterale per distribuzioni della famiglia esponenziale.
\item esercitazione: Intervalli di confidenza con ordinamento di Feldman Cousin per l'esponenziale e per una distribuzione triangolare; Richiamo del caso asintotico, LLR per gaussiana, distribuzione del $\chi^2$ e principali propriet\'a, caso asintotico per la poissoniana, confronto con gli intervalli centrali, e copertura.
\item esercitazione: Motivazioni per tests Locally Most Powerful. Test LMP generale unilaterale. Uso e proprieta' asintotiche del Likelihood-Ratio Test. Esempi di Tests ed esercizi svolti.
\item lezione: Concetto di "Goodness of fit" e sua utilit\'a. Tests di GOF. Definizione di P-value e suo uso in GOF test. Formula per la combinazione di 2 o pi\'u P-values. GOF di un fit con errori gaussiani con il LR, e test del "chi-quadrato" di Pearson.
\end{itemize}

%\end{verbatim}

\end{frame}

%\end{document}
