\section{Probability}

\begin{frame}{Assiomi probabilit\'a Kolmogorov}
\begin{itemize}
\item La probabilit\'a di un evento \'e un numero reale $\prob{(E)}\geq0$ per ogni $E\in S$ spazio campionario.
\item La probabilit\'a che almeno uno degli eventi del sample space occorra \'e 1.
\item $\sigma$-additivit\'a: per ogni sequenza numerabile di insiemi disgiunti (eventi mutualmente esclusivi)
\begin{equation*}
\prob{(\cup_{i=1}^{\infty}E_i)}=\sum_{i=1}^{\infty}\prob{(E_i)}
\end{equation*}
\end{itemize}
\begin{block}{Addition rule: probability that A or B will}
\begin{equation*}
\prob{(A\cup B)}=\prob{(A)}+\prob{(B)}-\prob{(A\cap B)}
\end{equation*}
\end{block}
\begin{block}{Variabile casuale}
Una variabile che assume uno specifico valore per ogni elemento dello spazio campionario S.
\end{block}
\begin{columns}[T]
\begin{column}{0.5\textwidth}

\begin{block}{Probabilit\'a condizionata}
\begin{equation*}
\prob{(B|A)}=\frac{\prob{(B\cap A)}}{\prob{(A)}}
\end{equation*}
\end{block}

\end{column}

\begin{column}{0.5\textwidth}

\begin{block}{Teorema di Bayes}
\begin{equation*}
\prob{(A|B)}=\frac{\prob{(B|A)}\prob{(A)}}{\prob{(B)}}
\end{equation*}
\end{block}

\end{column}
\end{columns}
\end{frame}

\begin{frame}{Interpretazioni di probabilit\'a}
\begin{columns}
\begin{column}{0.5\textwidth}

\begin{block}{Probabilit\'a frequentista}
Gli elementi di S sono risultati di una misura, un sottoinsieme A corrisponde a qualsiasi risultato nel sottoinsieme: A \'e detto evento. Se A contiene solo un elemento \'e detto evento elementare:
\begin{equation*}
\prob{(A)}=\lim_{n\to\infty}\frac{\text{number of occurrences of outcome A in n measurements}}{n}
\end{equation*}

\end{block}

\end{column}

\begin{column}{0.5\textwidth}

\begin{block}{Probabilit\'a bayesiana}



\end{block}

\end{column}

\end{columns}
\end{frame}