\section{Probability}

\begin{frame}{Assiomi probabilit\'a Kolmogorov}
\begin{itemize}
\item La probabilit\'a di un evento \'e $\prob{(E)}\geq0$ per ogni $E\in S$ spazio campionario.
\item La probabilit\'a degli eventi del sample space \'e 1.
\item $\sigma$-additivit\'a: per ogni sequenza numerabile di insiemi disgiunti (eventi mutualmente esclusivi)
\begin{equation*}
\prob{(\cup_{i=1}^{\infty}E_i)}=\sum_{i=1}^{\infty}\prob{(E_i)}
\end{equation*}
\end{itemize}
\begin{block}{Addition rule: probability that A or B will}
\begin{equation*}
\prob{(A\cup B)}=\prob{(A)}+\prob{(B)}-\prob{(A\cap B)}
\end{equation*}
\end{block}
\begin{block}{Variabile casuale}
Una variabile che assume uno specifico valore per ogni elemento dello spazio campionario S.
\end{block}
\begin{columns}[T]
\begin{column}{0.5\textwidth}
\begin{block}{Probabilit\'a condizionata}
\begin{equation*}
\prob{(B|A)}=\frac{\prob{(B\cap A)}}{\prob{(A)}}
\end{equation*}
\end{block}
\end{column}
\begin{column}{0.5\textwidth}
\begin{block}{Teorema di Bayes}
\begin{equation*}
\prob{(A|B)}=\frac{\prob{(B|A)}\prob{(A)}}{\prob{(B)}}
\end{equation*}
\end{block}
\end{column}
\end{columns}
\end{frame}

\begin{frame}{Interpretazioni di probabilit\'a}
\begin{block}{Probabilit\'a frequentista}
Gli elementi di S sono risultati di una misura, un sottoinsieme A corrisponde a qualsiasi risultato nel sottoinsieme: A \'e detto evento. Se A contiene solo un elemento \'e detto evento elementare:
\begin{equation*}
\prob{(A)}=\lim_{n\to\infty}\frac{\text{number of occurrences of outcome A in n measurements}}{n}
\end{equation*}
\end{block}
\begin{block}{Probabilit\'a bayesiana }
Gli elementi di A sono ipotesi o affermazioni:
\begin{equation*}
\prob{A}=\text{Degree of belief that A is true}
\end{equation*}
\begin{itemize}
\item Include relative frequency interpretation: The statement that a measure will yield a given outcome a certain fraction of times can be regarded as a hypothesis.
\item Subjective probability can be associated with value of unknown constants (mass of electron, etc): probability of $95\%$ that electron mass is in given interval.
\end{itemize}
\end{block}
\end{frame}

\section{Distribuzioni e variabili causuali}

\begin{frame}{Distribuzione di probabilit\'a di un osservabile. Valore di aspettazione, varianza e momenti}
Variabile casuale ($X$) campionata n volte ($x_i$) = Campione di dimensione n
\begin{itemize}
\item Distribuzione di probabilit\'a cumulante: $F(x)=P(x_n<x)$: PDF of X $f(x)=\TDy{x}{F(x)}$.
\item Distribuzione di probabilit\'a di $a(X)$ funzione di variabile casuale X con PDF $f(x)$: $g(a)=f(x(a))|\TDy{a}{x}|$
\item $E[X]=\sum x_iP(X=x_i)$.
\begin{align*}
&E[H(X)]=\intsinf{}H(x)f(x)\,dx
\end{align*}
\item Varianza $\var{(x_n)}=E[(x_n-\mu)^2]=\mu_2=\sigma^2(X)$.
\item momenti: $\mu_l=E[(X-\mu)^l]$
\end{itemize}
\end{frame}

\begin{frame}{Stimatori}
\begin{columns}[T]
\begin{column}{0.45\textwidth}
\begin{block}{Media aritmetica}
Sample mean from a group of observation is an estimate of sample $\mu$.
\end{block}
\end{column}
\begin{column}{0.55\textwidth}
\begin{block}{Gaussiana. T limite centrale.}
$\bar{x}_N=\frac{\sum_Nx_i}{N}$:
\begin{align*}
&\var{(\nmean{X})}=E[(\nmean{X}-\mu)^2]\\
&=E[(\frac{\sum x_i-\mu}{N})^2]=\frac{\sigma^2}{N}\\
&\lim_{N\to\infty}\prob{(|\nmean{X}-\mu|>\epsilon)}=0\\
&\nmean{Y}=\frac{\nmean{X}-\mu}{\sqrt{\var{\nmean{X}}}}=\frac{\sqrt{N}}{\sigma}(\nmean{X}-\mu)\\
&\exv{\nmean{Y}}=0,\ \var{\nmean{Y}}=1
\end{align*}
\end{block}
\end{column}
\end{columns}
\begin{block}{Covarianza di due variabili casuali}
\begin{align*}
&V_{xy}=E[(X-\mu_x)(Y-\mu_y)]=E[xy]-\mu_x\mu_y\\
&=\intsinf{}\intsinf{}xyf(x,y)\,dx\,dy-\mu_x\mu_y
\end{align*}
\end{block}
\end{frame}

\begin{frame}{Propagazione errori}
\begin{align*}
&Y(\vec{x})\approx+\sum_{i=1}^n[\PDy{x_i}{y}]|_{\vec{x}=\vec{\mu}}(x_i-\mu_i)\\
&E[Y(\vec{x})]\approx Y(\vec{\mu})\\
&E[Y^2(\vec{x})]\approx y^2(\vec{\mu})+\sum_{i,j=1}^n[\PDy{x_i}{y}\PDy{x_j}{y}]|_{\vec{x}=\vec{\mu}}V_{ij}\\
&\sigma^2_y\approx\sum_{i,j=1}^n[\PDy{x_i}{y}\PDy{x_j}{y}]|_{\vec{x}=\vec{\mu}}V_{ij}
\end{align*}
\end{frame}