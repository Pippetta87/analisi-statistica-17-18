\makeatletter
\let\@starttocorig\@starttoc
\makeatother%%

\documentclass[t,10pt,xcolor={usenames},fleqn]{beamer}

%%%Usefull link
%tikz-equations:
%http://www.wekaleamstudios.co.uk/posts/creating-a-presentation-with-latex-beamer-equations-and-tikz/

\hypersetup{pdfpagemode=FullScreen}

%% colors
\definecolor{NOW}{rgb}{1.0, 0.75, 0.0}
\definecolor{suspendend}{rgb}{0.76, 0.6, 0.42}
\definecolor{bittersweet}{rgb}{1.0, 0.44, 0.37}
\definecolor{brilliantlavender}{rgb}{0.96, 0.73, 1.0}
\definecolor{antiquefuchsia}{rgb}{0.57, 0.36, 0.51}
\definecolor{violetw}{rgb}{0.93, 0.51, 0.93}
\definecolor{Veronica}{rgb}{0.63, 0.36, 0.94}
\definecolor{atomictangerine}{rgb}{1.0, 0.6, 0.4}
\definecolor{darkgray}{rgb}{0.66, 0.66, 0.66}
\definecolor{brightcerulean}{rgb}{0.11, 0.67, 0.84}
\definecolor{cadmiumorange}{rgb}{0.93, 0.53, 0.18}
\definecolor{ochre}{rgb}{0.8, 0.47, 0.13}
\definecolor{midnightblue}{rgb}{0.1, 0.1, 0.44}
\definecolor{lemon}{rgb}{1.0, 0.97, 0.0}
\definecolor{grey}{rgb}{0.7, 0.75, 0.71}
\definecolor{amber}{rgb}{1.0, 0.75, 0.0}
\definecolor{almond}{rgb}{0.94, 0.87, 0.8}
\definecolor{bf}{RGB}{88, 86, 88}
\definecolor{bb}{RGB}{177, 177, 177}
\definecolor{keyword}{rgb}{0.25, 0.25, 0.28}
\definecolor{coolgrey}{rgb}{0.55, 0.57, 0.67}
\definecolor{todo}{rgb}{0.75, 0.0, 0.2}
\definecolor{must}{rgb}{1.0, 0.31, 0.0}
\definecolor{hp}{rgb}{0.0, 0.8, 0.6}

%%%%%%%%%%%%%%%%%%%%%%%%%%%%%%%%%%% importa pacchetti
\usepackage{usepkg}
%%
\renewbibmacro*{cite}{%
  \iffieldundef{shorthand}
    {\ifnameundef{labelname}
       {}
       {\printnames{labelname}%
        \setunit{\printdelim{nametitledelim}}}%
     \usebibmacro{cite:title}}%
    {\usebibmacro{cite:shorthand}}%
  \usebibmacro{cite:url}}

\newbibmacro{cite:url}{%
  \ifentrytype{online}
    {\setunit{\addspace}%
    \printfield{url}}
    {}}
    

%%%%%%%%%%%%%%%%%%%%%%%%%%%%%%%%%%% Funzioni generali
\usepackage{functions}
%http://tex.stackexchange.com/questions/246/when-should-i-use-input-vs-include
\newcommand{\setmuskip}[2]{#1=#2\relax} %%problem usinig mu with calc (req by mathtools) loaded
\usepackage{sources}
%\usepackage{length}
%%%%%%%%%%%%%%%%%%%%%%%%%%%%%%%%%%% Funzioni per questo file main
\usepackage{mathOp}
\usepackage{beamersetup}

\def\status{coazione}%ripetere
\def\keeptrying{coazione}
\usepackage{LocalF}
%%%%%%%%%%%%%%%%%%%%%%%%%%%%%%%%%


\title{Analisi statistica dei dati: 01 July 2020.}

% Let's get started
\begin{document}

\setcounter{tocdepth}{2}
%\let\mybibcat\noexpand\currfilebase
\tikzset{myscalar/.style={%%scale plot even cs axis: \def\myscale{}
		node distance=\myscale cm and \myscale cm,
		every node/.style={scale=\myscale},
		every axis/.style={
			major tick length=0.15*\myscale cm,
			minor tick length=0.1*\myscale cm,
			mark options={scale=\myscale},
			scale=\myscale
		}
	}
}
\begin{frame}
  \titlepage
\end{frame}

\begin{frame}[allowframebreaks]{TOC}
\tableofcontents[onlyparts]
\listofframes
\end{frame}
% Section and subsections will appear in the presentation overview
% and table of contents.
%\frame{\tableofcontents[onlyparts]}

\part{Intro}\linkdest{intro}
\begin{frame}{meta}
tolbf-listofframes; keyword-listofkeywords; todo; must
\end{frame}
\subfile{intro}
\subfile{reg}

\part{Succo}\linkdest{succo}
\subfile{succo}

\part{Variabili stocastiche e probabilit\'a}\linkdest{statdistro}
\subfile{probability}

\part{Inferenza statistica: stimatori puntuali e intervallari}\linkdest{inference}
%\begin{refsection}
%\nocite{*}
\subfile{estimators}
%\end{refsection}
%\begin{refsection}
%\nocite{*}
\part{Test statistici e Goodness-of-fit}
\subfile{testsgof}
%\end{refsection}
%\subfile{test}\linkdest{test}

\end{document}