\begin{frame}{Refs}
Introductory statistical inference with likelihood: basic statistical defs, stime, stime intervallari, hypothesis testing, maximum likelihood (cramer-crao), confidence interval, sufficienza 
likelihood: likelihood, likelihood ratio, sufficienza
\end{frame}

\section{Inferenza statistica: paradigmi}

\subsection{Frequentista}

\begin{frame}{Funzione di Likelihood}
\begin{block}{Funzione di likelihood}
\begin{columns}[T]
\begin{column}{0.5\textwidth}
\[L_{x_0}(m)=p(x_0;m)\]
x,y: eventi indipendenti:
\[L_{(x_0,y_0)}(m)=L_{x_0}(m)L_{y_0}(m)\]
\end{column}
\begin{column}{0.5\textwidth}
Definita a meno di costante moltiplicativa ($\log{L}$): invariante per cambiamento di osservabile (funzione solo dei dati)
\[L_{f(t_0)}(m)=|J(t_0)|L_{t_0}(m)\]
\end{column}
\end{columns}
\end{block}
\begin{block}{Likelihood per distro uniforme}
\pgfmathsetmacro{\unifM}{3}
\begin{columns}[T]
\begin{column}{0.5\textwidth}
\begin{tikzpicture}[scale=0.5,domain=0:1.5*\unifM]
\pgfmathsetmacro{\unifN}{(\unifM)^-1}
\begin{axis}[ylabel={$p(x;m)$},extra x ticks={\unifM}, extra x tick labels={$m$},
        extra x tick style={xticklabel style={yshift=-10}}]
%\draw[->] (-0.2,0) -- (1.5*\unifM,0) node[right] {$x$};
%\draw[->] (0,-0.2) -- (0,1.5*\unifM) node[above,red] {$p(x;m)$};
%\draw[color=red] plot[domain=0:\unifM, id=unif] function{\unifN} node[right] {$\frac{1}{m}$};
    \addplot[color=red] function [raw gnuplot, id=unifpdf, mark=none]{set xrange [0:\unifM]; plot \unifN};
\end{axis}
\end{tikzpicture}
\end{column}
\begin{column}{0.5\textwidth}
\begin{tikzpicture}[scale=0.5]
\begin{axis}[ylabel={$\mathcal{L}_{x_0}(m)$},extra x ticks={\data},extra x tick labels={$x_0$},
        extra x tick style={xticklabel style={yshift=-10}}]
\pgfmathsetmacro{\data}{(rand+1)/2*\unifM}
\pgfmathsetmacro{\unifN}{(\unifM)^-1}
    \addplot[color=orange] function [raw gnuplot, id=unifL, mark=none]{set xrange [\data:*]; plot 1/x};
\end{axis}
\end{tikzpicture}
\end{column}
\end{columns}
\end{block}
\end{frame}

\begin{frame}{Likelihood: bernoulli, ...}
\begin{columns}[T]
\begin{column}{0.5\textwidth}
    \begin{block}{Likelihood per distro uniforme}
\begin{align*}
&L_0=1-p\\
&L_1=p
\end{align*}
\end{block}
\end{column}
\begin{column}{0.5\textwidth}

\end{column}
\end{columns}
\end{frame}

\begin{frame}{Principio di likelihood}
    Se osservo x tutta informazione statistica in $L_x$: $L_x=L_y$ allora stesse inferenze da x e y.
\end{frame}