
\begin{frame}[allowframebreaks]{Reg Lez 18/19}\phantomsection\linkdest{rl18}
\cite{reg18}
\begin{itemize}
    \item 18/09/2018 - Concetto di \keyword{inferenza statistica} e sua relazione con le scienze sperimentali. Classificazione di tipi di inferenza. Concetto di \keyword{incertezza statistica e sistematica}.
    
    \item 19/09/2018 - \keyword{Probabilit\'a frequentista} (von Mises). Probabilit\'a soggettiva. Probabilit\'a matematica, sigma-algebre, assiomi di Kolmogorov. Teoremi elementari di probabilit\'a. Definizioni: Probabilit\'a congiunta, probabilit\'a condizionata, eventi indipendenti, \keyword{Teorema di Bayes}. Legge della Probabilit\'a Totale.
    
    \item 21/09/2018 (Esercizi) Probabilit\'a, probabilit\'a condizionata, assiomi di Kolmogorov. Esempi specifici su $P(A \cap B)$ e eventi indipendenti, concetto di indipendenza nel caso di pi\'u di due classi di eventi. Esempi con dadi al casino e indipendenza con mazzo da 52 e 51 carte.
    
    \item 25/09/2018 - \keyword{Osservabili e variabili aleatorie}. Distribuzioni di probabilit\'a (pmf) discrete e continue, densit\'a di probabilit\'a (pdf), Cumulanti (cdf). Distribuzioni congiunte, marginalizzazione, indipendenza. \keyword{Media campionaria e Legge dei Grandi Numeri}. Dimostrazione limitata di LLN tramite Disuguaglianza di Chebishev. Valore di aspettazione e sue propriet\'a. Varianza e Covarianza.
    
    \item 26/09/2018 - Esempio di uso del teorema di Bayes (esempio dei dadi); Propriet\'a dell'operatore di valore di aspettazione; $\var{}=E(x^2)-E(x)^2$; Calcolo della varianza di somma di due variabili aleatorie e introduzione della covarianza e della correlazione; Correlazione vs indipendenza (caso $y=x^2$), Uso della distribuzione cumulante ed esempio con $prob(max(x,y)=C)$ per x e y due lanci di dadi. Cambiamento di variabile in una o pi\'u dimensioni. Generazione di una esponenziale a partire da una uniforme, $y=x^2$, problema della lanciapalle da tennis (distr Cauchy).
    
    \item 28/09/2018 - Espansione dei \keyword{momenti di una statistica}:formule approssimate di uso comune(\keyword{propagazione errori}) e relativi pitfalls. Momenti di una Distribuzione. Momenti della distribuzione di Cauchy. Funzione Generatrice dei Momenti e sue propriet\'a. Esempio Uniforme. Funzione Caratteristica, e sue propriet\'a. Teorema di Paul Levy. Teorema del Limite Centrale. Distribuzioni Gaussiane.
    
    \item 02/10/2018 - \keyword{Processi di Bernoulli}. Derivazione frequentista della \keyword{distribuzione Binomiale}, e sua funzione Generatrice. D\keyword{istribuzione di Poisson} e sua Generatrice. Distribuzione Esponenziale (derivata dalla Poisson), e sua generatrice. Definizione di \keyword{Likelihood} e sua propriet\'a di base. Principio di Likelihood.
    
    \item 03/10/2018 - \keyword{Propagazione degli errori}, e caso della somma, prodotto e rapporto, con conto esplicito per due variabili indipendenti uniformi. Calcolo delle densità di probabilità nel caso di somma, prodotto e rapporto di due variabili aleatorie ($z=x+y$,$z=xy$,$z=x/y$), in generale, e per due uniformi. Discussione sulla validit\'a della propagazione degli errori per i rapporto di uniformi, e calcolo di $P(z)$ con ,$z=x/y$ rapporto di due Gaussiane. Propriet\'a della \keyword{distribuzione di Cauchy} (momenti, somma di due variabili di Cauchy e conseguenze per la media di Cauchy e sul teorema dei grandi numeri).
    
    \item 05/10/2018 - Derivazione della f. Caratteristica della Cauchy. Media di variabili Cauchy, e confronto con caso Gaussiano. Discussione del problema del processo esponenziale, importanza della scelta dell'Ensemble; \keyword{paradosso del Blue Bus} (suggerito per casa il calcolo della correlazione). Relazione con bias di selezione e trigger. Cenni alla Fallacia di Berkson, con esempi.
    
    \item 09/10/2018 - Uso della Likelihood a scopo inferenziale. Uso del teorema di Bayes in modo frequentista e in modo soggettivista. Probabilit\'a a priori, a posteriori, rapporto di likelihood, "betting odds". \keyword{Statistiche sufficienti}. \keyword{Teorema di Darmois}.
    
    \item 10/10/2018 - \keyword{Esercizi su inferenza bayesiana}: esercizio sul fascio di particelle (exe 1.3 dell’eserciziario del Cowan), esercizio sulla qualit\'a delle casse di munizioni, punto 1.1 e 1.2 del \keyword{Compito d’Esame 28/05/2018} (Dungeons$\&$Dragons). Breve riassunto della definizione di statistica sufficiente e esempi con la Gaussiana: la media aritmetica come statistica sufficiente per $\mu$, nota $\sigma^2$, e lo scarto quadratico medio da $\mu$ come statistica sufficiente per $\sigma^2$ con $\mu$ noto. Determinazione della \keyword{distribuzione di $x_{max}$} per la distribuzione uniforme $U[0,m]$ e dimostrazione tramite la definizione di statistica sufficiente che $x_{max}$ \'e una statistica sufficiente per il parametro m. (MICHAEL JOSEPH MORELLO)
    
    \item 16/10/2018 - \keyword{Informazione di Fisher}. Matrice Informazione. Score function. Additivit\'a, crescenza e linearit\'a dell'Informazione di Fisher. Esempi. Informazione da statistiche sufficienti. Riduzione della Informazione in mancanza di statistiche sufficienti: esempi accettanza finita (esponenziale troncato), istogrammazione dei dati.
    
    \item 17/10/2018 - Concetto di Stima Puntuale. \keyword{Stimatori}: consistenza, bias, varianza. \keyword{Disuguaglianza di Cramer-Rao}. Minimum Variance Bound e condizioni sotto le quali si raggiunge. Efficienza di uno Stimatore. Esempi svolti: esponenziale, media della gaussiana. Esempio $I_F$ non lineare: stima di upper limit di distribuzione uniforme usando media aritmetica vs $x_{max}$.
    
    \item 19/10/2018 - Breve riassunto e \keyword{applicazioni del Teorema di Darmois} con estrazione della relativa statistica sufficiente: Gaussiana e Binomiale. Calcolo dell'Informazione di Fischer per Poisson e Binomiale, rispettivamente per la media $\mu$ e per la probabilit\'a p. \keyword{Quando Cramer-Rao non vale}: verifica del fatto che la disuguaglianza di Cramer-Rao per $x_{max}$ da N-estrazioni $x_i$ da una distribuzione uniforme non vale. Studio della \keyword{statistica $x_{min}$}, come da problema 2 del compito di esame del 25/05/2018: calcolo della distribuzione di $x_{min}$ calcolo del valore di aspettazione, della varianza, nel limite finito e nel limite asintotico. Studio delle proprietà (consistenza e bias) di $x_{min}$ come estimatore di m (dove $\hat{m} = N \cdot x_{min}$). Dimostrazione che la statistica $x_{min}$ non \'e sufficiente per stimare m. Calcolo della informazione di Fischer di $x_{min}$ e verifica della validità o non validità della disuguaglianza di Cramer-Rao.
    
    \item 23/10/2018 - Metodi generali per la \keyword{costruzione di stimatori consistenti}. Metodo dei momenti, sue propriet\'a e applicazioni. Metodo degli stimatori impliciti. Stimatore di Massima Likelihood (MLE) e sue propriet\'a. Generalit\'a sul bias e metodi per la sua riduzione.
    
    \item \keyword{24/10/2018} - Considerazioni pratiche sul calcolo del MLE. MLE per istogrammi, schemi di binning, e confronto con la versione "unbinned". Stima numerica della varianza del MLE. Extended Likelihood vs. regular Likelihood. Limiti ad alta statistica e formule semplificate di uso comune. Stima puntuale con il Metodo dei Minimi Quadrati e sue proprieta'. Caso lineare e Teorema di Gauss-Markov.
    
    \item \keyword{26/10/2018} - Ripasso delle proprietà del MLE nel caso asintotico e nel caso finito per distribuzioni della famiglia esponenziale. MLE per la binomiale $\hat{p}$, e poissoniana $\hat{\mu}$ e studio delle loro proprietà. MLE della Gaussiana $\hat{mu}$ e $\hat{\sigma^2}$ e studio delle loro propriet\'a. MLE per la vita media di una distribuzione esponenziale. Sia nel caso $p(t,\tau)=1/\tau exp(-t/\tau)$ che nel caso $p(t,\lambda)=\lambda exp(-\lambda t)$ e studio delle loro propriet\'a. MLE per $x_{min}$ da distribuzione uniforme: valore di aspettazione, bias, varianza e verifica della non consistenza dell'estimatore. Per casa: studio del MLE per $x_{max}$.
    
    \item 30/10/2018 - \keyword{Outliers}. Concetto di Robustezza di uno stimatore. Stimatori di Location parameters. Stimatori da p-norme generalizzate. Stime di Moda, Mid-range, Mediana. Distribuzione e Propriet\'a della Mediana. Esempio mediana di distr. Cauchy.
    
    \item \keyword{31/10/2018} - Mediana: Massima LH per uniforme e comportamento asintotico, Esercizio dell'esame a risposta multipla (risposta random, e mediana), distribuzione di probabilit\'a per la mediana nel caso di pdf continua, limite asintotico per la mediana, calcolo esatto della , calcolo della varianza e limite per grandi N; mediana per distribuzione di Cauchy e discussione del MLE e dell'efficienza della mediana. Stimatori $L_{\alpha}$ e efficienze asintotiche.
    
    \item 06/11/2018 - Usi della Stima Puntuale e sue limitazioni. Esempi. Introduzione alla Stima Intervallare. Usi della Stima Intervallare. Principi di Stima Intervallare Bayesiana. Credibilit\'a. Costruzione di Regioni Credibili. Funzioni di ordinamento. Uso e vantaggi del Posterior Ordering. Esempi: Poisson, Uniforme. Introduzione ai concetti di Coverage e Confidence Level.
    
    \item 07/11/2018 -\keyword{QUEST: bayesian adaptive algorithm}. Algoritmo QUEST (\url{https://www.researchgate.net/publication/16355122_QUEST_A_Bayesian_adaptive_psychometric_method}): likelihood , Informazione di Fisher, metodo iterativo basato su approccio bayesiano, stimatore e terminazione dell'algoritmo in approccio frequentista, efficienza dello stimatore.
    
    \item 09/11/2018 - \keyword{Relazione tra Credibilit\'a media e Coverage media}. \keyword{Costruzione di Neyman}, \keyword{bande e regioni di confidenza}. Algoritmi di ordinamento, e propriet\'a del Probability Ordering. Problema del "flip-flopping". Possibilit\'a di regioni di confidenza vuote e discussione del suo significato. Uso del Likelihood-Ratio come funzione di ordinamento e sue propriet\'a. "Unified approach" di Feldman-Cousins. Concetto di "pivotal quantity". Il \keyword{LR come pivot asintotico} (Teorema di Wilks) e suo uso per la determinazione approssimata di Regioni di Confidenza.
    
    \item 13/11/2018 - Richiamo sugli intervalli bayesiani e frequentisti, e regole di ordinamento comune. - \keyword{Upper limit} bayesiano e frequentista per una poissoniana con 0 conteggi. - Costruzione dell'upper limit, lower limit e intervalli di confidenza con ordinamento di probabilit\'a e LR per la poissoniana. - Esercizio dei dadi di $D\&D$ dell'esame di maggio 2018.
    
    \item 14/11/2018 - Introduzione al concetto di \keyword{Test di Ipotesi}. Definizioni: Ipotesi Nulla, Ipotesi semplici e composte, regione critica, errori di tipo I e II. Problemi di Classificazione come Test di Ipotesi. Propriet\'a dei tests: potenza, consistenza, unbiasedness, MP, UMP. \keyword{Lemma di Neyman-Pearson}, con dimostrazione.
    
    \item 16/11/2018 - Test UMP unilaterale per distribuzioni della famiglia esponenziale. Motivazioni per \keyword{tests Locally Most Powerful}. Test LMP unilaterale di applicabilit\'a generale. \keyword{Likelihood-Ratio Test}: motivazione, uso e propriet\'a asintotiche. \keyword{Distribuzioni ''chi-quadro''}: definizioni e propriet\'a.
    
    \item 21/11/2018 - \keyword{Intervalli di confidenza frequentisti} sul parametro m della distribuzione $U[0,m]$ con singola misura x: limite superiore e inferiore con ordinamento crescente e decrescente su x, e estrazione degli intervalli di confidenza two-sided simmetrici e tramite l’algoritmo LR-ordering alla F-C. Intervalli di confidenza frequentisti sul parametro m della distribuzione Uniforme[0,m] con N estrazioni dell’osservabile x tramite la statistica sufficiente $x_{max}$: limite superiore con ordinamento crescente su $x_{max}$; intervalli two-sided tramite l’algoritmo di LR-ordering alla F-C; calcolo della distribuzione del $\lambda=-2\log{LR}$ e calcolo del relativo intervallo di confidenza.
    
    \item 23/11/2018 -\keyword{GOF}. Introduzione al concetto di "Goodness of fit". Tests di GOF e loro caratteristiche. Misure di GOF, definizione di p-value, unbiasedness. Uso di p-values in GOF test. Rischi di interpretazione, e critiche al concetto di p-value. Formule per la combinazione di 2, o di N p-values.
    
    \item 27/11/2018 - GOF per istogrammi con il LR. GOF di un fit con errori gaussiani con il LR, e test del "chi-quadrato" di Pearson. Problema del GOF per dati non binnati. \keyword{Test di Kolmogorov-Smirnov}.
    
    \item 28/11/2018 - Confronto degli \keyword{intervalli di confidenza frequentisti e bayesiani} sul parametro m della distribuzione $U[0,m]$ con N estrazioni, attraverso la statistica sufficiente $x_{max}$. Calcolo della funzione di Coverage e della funzione di Credibilità per entrami i casi e discussione sulla non validità del teorema dei valori medi $E[Cr(xmax)] = E[C(m)]$. Esempio di un test di Neyman-Pearson con ipotesi semplici, nel caso di due distribuzioni gaussiane con stessa varianza $\sigma^2$ ma differenti medie $\mu_0$ e $\mu_1$.
    
    \item  05/12/2018 Test di ipotesi: (esame 28/5/2018) Esercizio del \keyword{test per criticità con crescita esponenziale} di flusso di neutroni di una centrale nucleare, basato sugli ultimi N conteggi di flusso. Utilizzo del MLP test. Identificazione della PDF come della famiglia esponenziale e UMP test. Confronto delle due statistiche. Calcolo della significatività e del power in approssimazione asintotica per N grandi. Calcolo del MLE per piccoli valori della costante di crescita esponenziale. Introduzione del \keyword{Sign Test}. Calcolo della significatività e del power. Confronto del power con il test UMP su ipotesi semplici con due gaussiane con differente media.
    
    \item Mer 12/12/2018 - Concetto di "search" come combinazione di test+stima intervallare. Sensitivity region e sua ottimizzazione. Applicazione al counting experiment.
    \end{itemize}

\end{frame}

\begin{frame}[allowframebreaks]{Reg Lez 17/18}
%\begin{verbatim}

\begin{itemize}

\item lezione: Introduzione generale al corso. Elementi di Probabilita'. Probabilita' frequentista. Probabilit\'a soggettiva. Assiomi di Kolgomorov. Definizioni e identita' di base.

\item esercitazione: Esercizi su: probabilità, probabilit\'a condizionata, assiomi di Kolmogorov. Esempi specifici su $P(A \cap B)$ e eventi indipendenti.

\item esercitazione: Esercizi su: dadi al casino, esempi pratici si $P(A \cap B)$, concetto di indipendenza nel caso di più di due classi di eventi. Esempi di marginalizzazione di una funzione di distribuzione.

\item lezione: Definizioni e elementi base della Statistica. Statistiche, Valore di aspettazione, Osservabili, Indipendenza, Varianza e Covarianza. Distribuzioni di probabilita' discrete e continue, densita' di probabilita' (pdf), Cumulanti (cdf). Trasformazioni delle distribuzioni per cambiamento di variabile. Media campionaria e Legge dei Grandi Numeri.
 
\item esercitazione: Correlazione vs indipendenza (caso $y=x^2$), uso della distribuzione cumulante. Cambiamento di variabile in una o più dimensioni. Calcolo delle densità di probabilità nel caso di somma, prodotto e rapporto di due variabili aleatorie ($z=x+y,z=xy,z=x/y$).

\item lezione: Momenti di una Distribuzione. Approssimazione dei momenti di una statistica ("propagazione errori"). Funzione Generatrice dei Momenti. Esempi. Processi di Bernoulli. Distribuzione Binomiale. Distribuione di Poisson e distribuzione (Esponenziale) della distanza tra i suoi eventi. Funzione Caratteristica.

\item esercitazione: Introduzione alla binominale e alla poissoniana, calcolo delle funzioni generatrici dei momenti, delle medie e delle varianze - Plot e tabelle di probabilità delle distribuzioni per alcuni casi specifici - Esempio (Binomiale): Calcolo di media e varianza per la variabile di asimmetria $A=U-D/U+D$ e per $eff=k/N$ - Esempio (Poisson) uso della poissoniana e calcolo della PDF per i conteggi in un contatore inefficiente. - Esercizio per casa: Calcolare la PDF per A e per $eff=k/N$ e negative binomial - Introduzione alla distribuzione esponenziale e Cauchy, per exp. calcolo della funzione caratteristica , della media e della varianza. - Esempio: Prescale stocastico e deterministico e fun. caratteristiche - Esercizio per casa: calcolo della fun. caratteristica per Cauchy

\item lezione: Teorema del Limite Centrale e densita' di probabilita' Gaussiana. Definizione e proprieta' generali della Funzione di Likelihood. Esempi. Uso della Likelihood in inferenze di tipo Bayesiano: prior, posterior, betting odds, belief-updating ratio. Cenni su effetti di incertezza sistematica.

\item esercitazione: esempi di inferenza bayesiana: problema della meningite, problema del fascio di particelle, problema delle casse di munizioni, problema del numero del taxi e della temperatura. Esempi e criticità dell'uso di una prior "impropria", come il caso della distribuzione di poisson con conteggio nullo.

--per 16 Agosto--
\phantomsection{}\hypertarget{datareduction}{}

\item lezione: Concetto di "riduzione dei dati". Statistiche sufficienti, e statistiche sufficienti minimali. Esempi Uniforme, Esponenziale, Gaussiano. Teorema di Darmois.
    
\item lezione: Esempio e dimostrazione sufficienza per la statistica $max(x)$ della $U(0,m)$. Definizione di Matrice di Informazione di Fisher. Additivita', crescenza e linearita' dell'Informazione di Fisher. Esempi. Informazione da statistiche sufficienti. Perdita di Informazione in mancanza di statistiche sufficienti: esempi istogrammazione dei dati, accettanza finita (esponenziale troncato).

\item esercitazione: - Informazione di Fisher per esponenziali troncate e informazioni sulle code dell’esponenziale; - Informazione e statistiche sufficienti (il caso della distribuzione uniforme); - Statistiche sufficienti (gaussiana) - Teorema di Darmois (gaussiana, poisson) - Informazione di Fisher (gaussiana, poisson e risultati per binomiale)

\item lezione: Concetto di Stima Puntuale. Stimatori: consistenza, bias, varianza. Disuguaglianza di Cramer-Rao. Minimum Variance Bound e condizioni sotto le quali si raggiunge. Efficienza di uno Stimatore.

\item lezione: Metodi generali per la costruzione di stimatori consistenti. Metodo dei momenti e sue applicazioni. Stimatore di Massima Likelihood (MLE) e sue propriet\'a. Esempio stima simultanea di media e varianza di Gaussiana. Metodi di correzione del bias.

\item esercitazione: Stimatore di massima likelihood e sue proprietà (consistenza, calcolo del valore atteso e della varianza, bias e limiti asintotici) per misure di variabili aleatorie per distribuzioni binomiale, poissoniana, esponenziale, uniforme. Media pesata e sua varianza.

\item lezione: MLE per istogrammi. Extended Likelihood vs. regular Likelihood. Casi particolari. Limiti ad alta statistica e formule semplificate di uso comune. Metodo dei Minimi Quadrati. Consistenza e unbiasedness. Caso lineare e Teorema di Gauss-Markov.

\item lezione: Outliers. Concetto di Robustezza di uno stimatore. Stimatori di Location parameters. Stimatori da p-norme generalizzate. Stime di Moda, Mid-range, Mediana. Distribuzione e Proprieta' della Mediana. Esempio mediana di distr. Cauchy.

\phantomsection{}\hypertarget{stimatori}{}

\item lezione: Usi della Stima Puntuale e sue limitazioni. Esempi. Usi della Stima Intervallare. Introduzione alla Stima Intervallare. Principi di Stima Intervallare Bayesiana. Credibilita'. Costruzione di Regioni Credibili. Funzioni di ordinamento. Uso e vantaggi del Posterior Ordering. Esempi: Poisson, Uniforme.

-- Per 17 Agosto --

\phantomsection{}\hypertarget{confidence}{}

\item lezione: Introduzione ai concetti di Coverage e Confidence Level. Proprieta', similarita' e differenze con il concetto di Credibility. Costruzione di Neyman, bande di confidenza. Algoritmi di ordinamento, e proprieta' del Probability Ordering. Esempi: Poisson upper limits, Uniforme (N=1,2); confronto con corrispondenti risultati Bayesiani. Relazione tra Credibilita' media e Coverage media.

\item  lezione: Regioni di Confidenza: problema del "flip-flopping". Possibilit\'a di regioni di confidenza vuote e discussione del suo significato. Uso del LIkelihood-Ratio come funzione di ordinamento e sue proprieta'. "Unified approach" di Feldman-Cousins. Concetto di "pivotal quantity". Il LR come pivot asintotico (Teorema di Wilks) e suo uso per la determinazione approssimata di Regioni di Confidenza. La distribuzione del Chi-quadro e sua interpretazione. 

\phantomsection{}\hypertarget{tests}{}

\item  lezione: Introduzione al concetto di Test di Ipotesi. Definizioni: Ipotesi Nulla, Ipotesi semplici e composte, regione critica, errori di tipo I e II. Problemi di Classificazione visti come Test di Ipotesi. Proprieta' dei tests: potenza, consistenza, unbiasedness, MP, UMP. Lemma di Neyman-Pearson, con dimostrazione. Esistenza di test UMP unilaterale per distribuzioni della famiglia esponenziale.

\item esercitazione: Intervalli di confidenza con ordinamento di Feldman Cousin per l'esponenziale e per una distribuzione triangolare; Richiamo del caso asintotico, LLR per gaussiana, distribuzione del $\chi^2$ e principali propriet\'a, caso asintotico per la poissoniana, confronto con gli intervalli centrali, e copertura.

\item esercitazione: Motivazioni per tests Locally Most Powerful. Test LMP generale unilaterale. Uso e proprieta' asintotiche del Likelihood-Ratio Test. Esempi di Tests ed esercizi svolti.


\item lezione: Concetto di "Goodness of fit" e sua utilit\'a. Tests di GOF. Definizione di P-value e suo uso in GOF test. Formula per la combinazione di 2 o pi\'u P-values. GOF di un fit con errori gaussiani con il LR, e test del "chi-quadrato" di Pearson.\phantomsection{}\hypertarget{goodnessoffit}{}

--Per 18 Agosto --
\end{itemize}

%\end{verbatim}

\end{frame}